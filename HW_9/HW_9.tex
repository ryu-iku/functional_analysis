\documentclass{article}
\usepackage{amsmath, amssymb, amsthm}
\usepackage{geometry}
\geometry{a4paper, margin=1in}

\DeclareMathOperator{\esssup}{ess\,sup}
\newcommand{\norm}[1]{\left\|#1\right\|}
\newcommand{\abs}[1]{\left|#1\right|}
\newcommand{\set}[1]{\left\{#1\right\}}
\newcommand{\R}{\mathbb{R}}
\newcommand{\C}{\mathbb{C}}
\newcommand{\N}{\mathbb{N}}
\newcommand{\ellp}{\ell^p}
\newcommand{\ellq}{\ell^q}
\newcommand{\ip}[2]{\left\langle #1, #2 \right\rangle}
\newcommand{\re}{\operatorname{Re}}
\newcommand{\dist}{\operatorname{dist}}
\newcommand{\Res}{\operatorname{Res}}

\newtheorem{problem}{Problem}

\begin{document}

\begin{center}
    \textbf{\Large Functional Analysis - Homework 9}
\end{center}

\hrule
\vspace{0.5em}

% ---------------------------
% Problem 1
% ---------------------------
\begin{problem}
Let $H$ be a Hilbert space and let $K\subset H$ be a non-empty closed convex set.
\begin{enumerate}
    \item[a)] Prove that for every $f\in H$ there exists an element $u\in K$ such that $\norm{f-u}=\min_{v\in K}\norm{f-v}=\text{dist}(f,K)$.
    \item[b)] Prove that (a) is equivalent to the property: $\re\ip{f-u}{v-u} \le 0 \quad \text{for all } v\in K$.
    \item[c)] Use now (b) to prove the uniqueness of such $u$.
    \item[d)] Prove that $P_{K}$ is contractive, i.e., $\norm{P_{K}f_{1}-P_{K}f_{2}}\le\norm{f_{1}-f_{2}}$.
\end{enumerate}
\end{problem}
% ------------- Solution
\subsection*{a)}

Let $(v_n)_{n \in \N} \subset K$ be a minimizing sequence such that $\lim_{n \to \infty} \norm{f - v_n} = \dist(f, K)$.

With the parallelogram equality, for $m, k \in \N$ we have
\begin{align}
    \norm{v_m - v_k}^2
    &= \norm{(f - v_m) - (f - v_k)}^2 \\
    &= 2 \norm{f- v_m}^2 + 2 \norm{f - v_k}^2 - \norm{(f - v_m) + (f - v_k)}^2 \\
    &= 2 \norm{f- v_m}^2 + 2 \norm{f - v_k}^2 - 4 \norm{f - \frac{v_m + v_k}{2}}^2
\end{align}
Since $K$ is convex, $\frac{v_m + v_k}{2} \in K$, and thus $\norm{f - \frac{v_m + v_k}{2}} \ge \dist(f, K)$.
\begin{align}
    \norm{v_m - v_k}^2
    &\le 2 \norm{f- v_m}^2 + 2 \norm{f - v_k}^2 - 4 \dist(f, K)^2 \\
    &= 2 \left( \norm{f- v_m}^2 - \dist(f, K)^2 \right) + 2 \left( \norm{f- v_k}^2 - \dist(f, K)^2 \right)
\end{align}
Since $\norm{f - v_n} \to \dist(f, K)$, the right-hand side tends to $0$ as $m, k \to \infty$. This implies $(v_n)$ is Cauchy. Because $H$ is complete and $K$ is closed, $(v_n)$ converges to some $u \in K$ which satisfies $\norm{f - u} = \dist(f, K)$.

\subsection*{b)}

Let $\mathbf{(1)}$ be $\norm{f-u} = \min_{v \in K}\norm{f-v}$ and $\mathbf{(2)}$ be $\re\ip{f-u}{v-u} \le 0$ for all $v \in K$.

\paragraph{$\mathbf{(1) \implies (2)}$}
Assume $u$ satisfies (1). For arbitrary $v \in K$, let $v_t = (1-t)u + tv \in K$ for $t \in [0, 1]$. By the minimization property, $\norm{f-u}^2 \le \norm{f-v_t}^2$.
\begin{align}
    \norm{f-u}^2
    &\le \norm{(f-u) - t(v-u)}^2 \\
    &= \norm{f-u}^2 - 2t \re\ip{f-u}{v-u} + t^2\norm{v-u}^2 \\
    \therefore
    2t \re\ip{f-u}{v-u}
    &\le t^2\norm{v-u}^2
\end{align}
Dividing by $2t$ for $t > 0$ and taking the limit as $t \to 0^+$ yields $\re\ip{f-u}{v-u} \le 0$.

\paragraph{$\mathbf{(2) \implies (1)}$}
Assume $u$ satisfies (2). For any $v \in K$, we expand $\norm{f-v}^2$:
\begin{align}
    \norm{f-v}^2 &= \norm{(f-u) - (v-u)}^2 \\
    &= \norm{f-u}^2 - 2\re\ip{f-u}{v-u} + \norm{v-u}^2
\end{align}
Since $\re\ip{f-u}{v-u} \le 0$ by (2), the term $-2\re\ip{f-u}{v-u} \ge 0$.
\begin{align}
    \norm{f-v}^2 \ge \norm{f-u}^2 + \norm{v-u}^2
\end{align}
Since $\norm{v-u}^2 \ge 0$, we conclude that $\norm{f-v} \ge \norm{f-u}$, proving (1).

\subsection*{c)}
Assume $u_1$ and $u_2$ are both minimizers in $K$. By part (b), they must satisfy the inequality (2).
\begin{itemize}
    \item $u_1$ is the projection of $f$. Setting $v=u_2$: $\re\ip{f-u_1}{u_2-u_1} \le 0$.
    \item $u_2$ is the projection of $f$. Setting $v=u_1$: $\re\ip{f-u_2}{u_1-u_2} \le 0$.
\end{itemize}
Using $u_1-u_2 = -(u_2-u_1)$, the second inequality is equivalent to $\re\ip{f-u_2}{u_2-u_1} \ge 0$. Summing the two relations:
\begin{align}
    0 &\ge \re\ip{f-u_1}{u_2-u_1} - \re\ip{f-u_2}{u_2-u_1} \\
    0 &\ge \re\ip{(f-u_1) - (f-u_2)}{u_2-u_1} \\
    0 &\ge \re\ip{u_2-u_1}{u_2-u_1} \\
    \norm{u_2-u_1}^2 &\le 0
\end{align}
Since the norm squared must be non-negative, $\norm{u_2-u_1}^2 = 0$, which implies $u_1 = u_2$.

\subsection*{d)}
Let $u_1 = P_K f_1$ and $u_2 = P_K f_2$. With property (b):
\begin{itemize}
    \item For $f_1$ and $u_1$: set $v=u_2$. $\re\ip{f_1-u_1}{u_2-u_1} \le 0$.
    \item For $f_2$ and $u_2$: set $v=u_1$. $\re\ip{f_2-u_2}{u_1-u_2} \le 0$, which implies $-\re\ip{f_2-u_2}{u_2-u_1} \le 0$.
\end{itemize}
Summing these two inequalities:
\begin{align}
    0 &\ge \re\ip{f_1-u_1}{u_2-u_1} - \re\ip{f_2-u_2}{u_2-u_1} \\
    0 &\ge \re\ip{(f_1-u_1) - (f_2-u_2)}{u_2-u_1} \\
    0 &\ge \re\ip{(f_1-f_2) - (u_1-u_2)}{u_2-u_1} \\
    0 &\ge \re\ip{f_1-f_2}{u_2-u_1} - \re\ip{u_1-u_2}{u_2-u_1} \\
    \norm{u_2-u_1}^2 &\le \re\ip{f_1-f_2}{u_2-u_1}
    \le \abs{\ip{f_1-f_2}{u_2-u_1}}
\end{align}
Applying the Cauchy-Schwarz inequality, $\abs{\ip{x}{y}} \le \norm{x}\norm{y}$:
\begin{align}
    \norm{u_1-u_2}^2 &\le \norm{f_1-f_2}\norm{u_2-u_1}
\end{align}
Dividing by $\norm{u_1-u_2}$ (assuming it is non-zero) yields the desired result: $\norm{u_1-u_2} \le \norm{f_1-f_2}$.

\vspace{0.5em}
\hrule
\vspace{0.5em}

% ---------------------------
% Problem 2
% ---------------------------

\begin{problem}
\end{problem}
% ------------- Solution
With $\overline{x-i} = x+i$ and $\overline{x+i} = x-i$,
\begin{align}
    \ip{f_m}{f_n} &= \int_{-\infty}^{\infty} f_m(x)\overline{f_n(x)} dx \\
    &= \frac{1}{\pi} \int_{-\infty}^{\infty} \frac{(x-i)^{m}}{(x+i)^{m+1}} \frac{(x+i)^{n}}{(x-i)^{n+1}} dx \\
    &= \frac{1}{\pi} \int_{-\infty}^{\infty} \frac{(x-i)^{m-n-1}}{(x+i)^{m-n+1}} dx
\end{align}
Let $g(z) = \frac{1}{\pi} \frac{(z-i)^{m-n-1}}{(z+i)^{m-n+1}}$. We use the Residue Theorem on a semi-circular contour in the upper half-plane. The integral is equal to $2\pi i \sum \Res(g, z_k)$, where $z_k$ are poles in the upper half-plane, and the poles are at $z=\pm i$.

\paragraph{Case 1: $m=n$}
The integrand simplifies to $g(z) = \frac{1}{\pi} \frac{(z-i)^{-1}}{(z+i)^{1}} = \frac{1}{\pi(z-i)(z+i)}$.
The only pole in the upper half-plane is a simple pole at $z=i$.
\begin{align}
    \Res(g, i) &= \lim_{z \to i} (z-i) g(z) = \lim_{z \to i} \frac{1}{\pi(z+i)} = \frac{1}{\pi(2i)} \\
    \ip{f_n}{f_n} &= 2\pi i \cdot \Res(g, i) = 2\pi i \cdot \frac{1}{2\pi i} = 1
\end{align}

\paragraph{Case 2: $m \ne n$}

\textbf{1) $m > n$}
Let $k = m-n \ge 1$. The integrand is $g(z) = \frac{1}{\pi} \frac{(z-i)^{k-1}}{(z+i)^{k+1}}$.
The only singularity is a pole at $z = -i$ of order $k+1$. Since this pole is in the lower half-plane, the upper half-plane contour encloses no singularities.
\begin{align}
    \ip{f_m}{f_n} = 2\pi i \cdot (0) = 0
\end{align}

\textbf{2) $m < n$}
Let $k = n-m \ge 1$. The integrand is $g(z) = \frac{1}{\pi} \frac{(z+i)^{k-1}}{(z-i)^{k+1}}$.
The only singularity is a pole at $z = i$ of order $p=k+1$, which is in the upper half-plane.
The residue is:
\begin{align}
    \Res(g, i) &= \frac{1}{k!} \lim_{z\to i} \frac{d^{k}}{dz^{k}} \left[ (z-i)^{k+1} g(z) \right] \\
    &= \frac{1}{k!} \lim_{z\to i} \frac{d^{k}}{dz^{k}} \left[ \frac{1}{\pi} (z+i)^{k-1} \right]
\end{align}
Since $k \ge 1$, the $k$-th derivative of the polynomial $(z+i)^{k-1}$ (which has degree $k-1$) is zero.
\begin{align}
    \frac{d^{k}}{dz^{k}} (z+i)^{k-1} &= 0 \\
    \therefore \Res(g, i) &= 0
\end{align}
\begin{align}
    \ip{f_m}{f_n} &= 2\pi i \cdot \Res(g, i) = 0
\end{align}

Therefore, the functions satisfy $\ip{f_m}{f_n} = 1$ if $m=n$ and $0$ if $m\ne n$, thus they form an orthonormal set.

\end{document}