\documentclass{article}
\usepackage{amsmath, amssymb, amsthm}
\usepackage{geometry}
\geometry{a4paper, margin=1in}

\DeclareMathOperator{\esssup}{ess\,sup}
\newcommand{\norm}[1]{\left\|#1\right\|}
\newcommand{\abs}[1]{\left|#1\right|}
\newcommand{\set}[1]{\left\{#1\right\}}
\newcommand{\R}{\mathbb{R}}
\newcommand{\C}{\mathbb{C}}
\newcommand{\N}{\mathbb{N}}
\newcommand{\ellp}{\ell^p}
\newcommand{\ellq}{\ell^q}
\newcommand{\ip}[2]{\left\langle #1, #2 \right\rangle}

\newtheorem{problem}{Problem}

\begin{document}

\begin{center}
    \textbf{\Large Functional Analysis - Homework 5}
\end{center}

\hrule
\vspace{0.5em}

% ---------------------------
% Problem 1
% ---------------------------
\begin{problem}
\end{problem}
% ------------- Solution
\begin{proof}
\textbf{a)}
First, we prove $M$ is continuous at $(0, 0)$, which is equivalent to
\begin{gather}
\forall \epsilon > 0, \exists \delta > 0, \text{ s.t. }
\norm{(x,y)}_{X \times Y} < \delta
\implies
\norm{M(x, y)}_Z < \epsilon
\end{gather}

Suppose
\begin{gather}
    \norm{(x,y)}_{X \times Y} 
    = \norm{x}_X + \norm{y}_Y < \delta = \sqrt{\frac{\epsilon}{C}}
\end{gather}

, we will have
\begin{gather}
    \norm{M(x, y)}_Z 
    \leq C \norm{x}_X \norm{y}_Y
    \\
    \leq C (\norm{x}_X + \norm{y}_Y) (\norm{x}_X + \norm{y}_Y)
    \\
    = C \delta^2 = \epsilon
\end{gather}
, which is exactly equation (1). Thus, $M$ is continuous at $(0, 0)$.
Since $M$ is bi-linear, the continuity holds for the whole space.

\textbf{b)}

Assume $X$ is complete, which means it is Banach. 

Define a linear opertor
\begin{gather}
    L_y: X \to Z, L_y (x) := M(x, y)
\end{gather}
with $x \in X, y \in Y$.

Since $ y \mapsto M(x',y) $ is continuous and linear, $\norm{M(x',y)}$ should be bounded with fixed $x' \in X$. Thus,
\begin{gather}
    \sup_{\norm{y}_Y=1} \norm{L_y(x')}_Z
    = C_{x'}
    < \infty
\end{gather}
, which implies pointwise boundedness of $L_y$ for $\{y \in Y : \norm{y}_Y=1\}$.

With Uniform Boundedness Theorem, we have the operator boundedness:
\begin{gather}
    \sup_{\norm{y}_Y=1} \norm{L_y}
    = C
    < \infty
\end{gather}

\begin{align}
    \because L_y(x)
    &= M(x, y)
    \\
    &= \norm{y}_Y M(x, \frac{y}{\norm{y}_Y})
    \\
    &= \norm{y}_Y M(x, u),
    \text{ where } 
    u = \frac{y}{\norm{y}_Y}
    \\
    &= \norm{y}_Y L_{u}(x)
    \text{, where } 
    \norm{u} = 1
    \\
    \therefore \norm{L_y}
    &\leq \norm{y}_Y C
\end{align}

Since $ x \mapsto M(x,y')$ is continuous and linear, $\norm{M(x,y')}$ is bounded with fixed $y' \in Y$.
\begin{align}
    \norm{M(x,y')}_Z
    = 
    \norm{L_{y'}(x)}_Z
    \leq \norm{L_{y'}} \norm{x}_X
    \\
    \therefore
    \norm{M(x, y)}_Z 
    \leq C \norm{x}_X \norm{y}_Y
\end{align}

\end{proof}

\hrule
\vspace{0.5em}

% ---------------------------
% Problem 2
% ---------------------------

\begin{problem}
\end{problem}
% ------------- Solution
\begin{proof}
\textbf{(1) continuity $\implies$ closedness }

Suppose $f$ is continuous. 
With a closed set $V$ in $R$, $f^{-1}(V)$ should be closed in $E$.

Therefore, as $\{\alpha \}$ is closed in $R$, $H_{\alpha} = f^{-1}(\{\alpha \})$ is closed in $E$.

\textbf{(2) closedness $\implies$ continuity } 

Suppose there exists a quotient space $E / H_0$, with a norm
\begin{align}
    \norm{[x]}_{E / H_0} = \inf_{y \in H_0} \norm{x - y}_E
\end{align}
, with $x \in E$ and $[x]$ is the equivalence class of x.

Verify the norm as follows.

1. Homogeneity:
$\forall \lambda \in \R, x \in E$,
\begin{align}
    \norm{[\lambda x]}_{E / H_0} 
    &= \inf_{y \in H_0} \norm{\lambda x - y}_E
    \\
    &= \inf_{y' \in H_0} \norm{\lambda x - \lambda y'}_E
    \text{, where } y' = \frac{y'}{\lambda}
    \\
    &= \inf_{y' \in H_0} \lambda \norm{x - y'}_E
    \\
    &= \lambda \norm{[x]}_{E / H_0} 
\end{align}

2. triangle inequality:
$\forall \lambda \in \R, x_1, x_2 \in E$,
\begin{align}
    \norm{[x_1 + x_2]}_{E / H_0} 
    &= \inf_{y \in H_0} \norm{ x_1 + x_2 - y}_E
    \\
    &= \inf_{y' \in H_0} \norm{ x_1 -  y'}_E + \inf_{y' \in H_0} \norm{ x_2 -  y'}_E
    \text{, where } y' = \frac{y}{2}
    \\
    &= \norm{[x_1]}_{E / H_0} + \norm{[x_2]}_{E / H_0}
\end{align}
.

3. Positive definiteness:
\begin{align}
    &\norm{[x]}_{E / H_0} = 0
    \\
    &\inf_{y \in H_0} \norm{ x - y}_E = 0
    \\
    &\iff x \in H_0
    \text{, since $H_0$ is closed}
    \\
    &\iff [x] = H_0
\end{align}

Therefore, the norm exists.


Define a function
\begin{align}
    g: E / H_0 \to H, g([x]) := f(x)
\end{align}
, with $x \in E$.

Let $y \in H_0$ be the closest point to $x \in E$, $ \norm{x - y}_E = \inf_{y \in H_0} \norm{x - y}_E$.
\begin{align}
    \therefore \abs{g([x])} 
    &= \abs{f(x)}
    \\
    &= \abs{f(x) - f(y)},
    \text{ since } f(y) = 0
    \\
    &= \abs{f(x - y)}
    \\
    &\leq \norm{f} \norm{x -y}_E
    \\
    &= \norm{f} \inf_{y \in H_0} \norm{x - y}_E
    \\
    &= \norm{f} \norm{[x]}_{E / H_0}
\end{align}
, which implies $g$ is bounded, and in turn it is continuous.

With the same variables, we can have
\begin{align}
    \norm{[x]}_{E / H_0}
    &= \inf_{y \in H_0} \norm{x - y}_E
    \\
    \norm{x - 0}_E,
    \text{ since } f(0) = 0
    \\
    \norm{x}_E
\end{align}
. Thus, the canonical projection map $h: E \to E / H_0, h(x) = [x]$ is also continuous.

Therefore, $f = g \circ h$ is continuous.

\end{proof}

\hrule
\vspace{0.5em}

% ---------------------------
% Problem 3
% ---------------------------

\begin{problem}
\end{problem}
% ------------- Solution
\begin{proof}
    The norm $\norm{\cdot}_2$ is equivalent to $\norm{\cdot}_1$ on $Y$, meaning there exist constants $c, C > 0$ such that $\forall y \in Y$
    \begin{align}
    c \norm{y}_1 \leq \norm{y}_2 \leq C \norm{y}_1
    \end{align}

    We define a new norm $\norm{\cdot}$ on $X$
    \begin{align}
    \norm{x} = \inf_{y \in Y} \left( \norm{x-y}_1 + \norm{y}_2 \right) \quad \forall x \in X
    \end{align}

    \textbf{1. Restriction to $Y$ is equivalent to $\norm{\cdot}_2$:}
    For $x \in Y$, taking $y=x$ in the infimum gives $\norm{x} \le \norm{x-x}_1 + \norm{x}_2 = \norm{x}_2$.
    For the other direction, since $x \in Y$ and $y \in Y$, $x-y \in Y$. Using the triangle inequality for $\norm{\cdot}_2$ and the equivalence bounds shows $\norm{x} \ge K \norm{x}_2$ for some $K>0$. Thus, $\norm{\cdot}|_Y$ is equivalent to $\norm{\cdot}_2$.

    \textbf{2. Equivalence on $X$:}
    We show $c \norm{x}_1 \le \norm{x} \le \norm{x}_1$.

    - Upper bound ($\norm{x} \le \norm{x}_1$):
    Choosing $y=0 \in Y$ in the definition.
    \begin{align}
        \norm{x} \le \norm{x-0}_1 + \norm{0}_2 = \norm{x}_1
    \end{align}

    - Lower bound ($\norm{x} \ge c \norm{x}_1$):
    For any $y \in Y$, we use the triangle inequality for $\norm{\cdot}_1$ and the equivalence $\norm{y}_1 \le c^{-1} \norm{y}_2$ on $Y$.
    \begin{align}
        \norm{x}_1 = \norm{x-y+y}_1 \le \norm{x-y}_1 + \norm{y}_1 \le \norm{x-y}_1 + c^{-1} \norm{y}_2
    \end{align}

    Multiplying by $c$ (and noting that $c \le 1$ can be assumed without loss of generality, but $c$ is a constant, so we proceed):
    \begin{align}
    c \norm{x}_1 \le c \norm{x-y}_1 + \norm{y}_2 \le \norm{x-y}_1 + \norm{y}_2
    \end{align}

    Taking the infimum over $y \in Y$ gives:
    \begin{align}
    c \norm{x}_1 \le \inf_{y \in Y} \left( \norm{x-y}_1 + \norm{y}_2 \right) = \norm{x}
    \end{align}
    
    Since $c \norm{x}_1 \le \norm{x} \le \norm{x}_1$, the norm $\norm{\cdot}$ is equivalent to $\norm{\cdot}_1$ on $X$, and its restriction to $Y$ is equivalent to $\norm{\cdot}_2$.
\end{proof}


\hrule
\vspace{0.5em}



\end{document}