\documentclass{article}
\usepackage{amsmath, amssymb, amsthm}
\usepackage{geometry}
\geometry{a4paper, margin=1in}

\DeclareMathOperator{\esssup}{ess\,sup}
\newcommand{\norm}[1]{\left\|#1\right\|}
\newcommand{\abs}[1]{\left|#1\right|}
\newcommand{\set}[1]{\left\{#1\right\}}
\newcommand{\R}{\mathbb{R}}
\newcommand{\C}{\mathbb{C}}
\newcommand{\N}{\mathbb{N}}
\newcommand{\ellp}{\ell^p}
\newcommand{\ellq}{\ell^q}
\newcommand{\ip}[2]{\left\langle #1, #2 \right\rangle}

\newtheorem{problem}{Problem}

\begin{document}

\begin{center}
    \textbf{\Large Functional Analysis - Homework 7}
\end{center}

\hrule
\vspace{0.5em}

% ---------------------------
% Problem 1
% ---------------------------
\begin{problem}

    
\end{problem}
% ------------- Solution
Let $(X,\|\cdot\|_{X})$ be a Banach space and $X^{*}$ its dual space. The sequence $(f_{n})_{n\in\mathbb{N}} \subset X^{*}$ and $f \in X^{*}$. We denote weak* convergence by $f_{n} \xrightarrow{*} f$.

\subsection*{a)}
Statement: $f_{n} \xrightarrow{*} f$ if and only if $f_{n}(x)\rightarrow f(x)$ for every $x\in X$.

\begin{proof}
    ($\Rightarrow$) 
    
    Assume $f_{n} \xrightarrow{*} f$.
    By definition, $f_{n} \xrightarrow{*} f$ means $f_n$ converges to $f$ in the weak* topology $\sigma(X^{*}, X)$. The weak* topology is the coarsest topology on $X^{*}$ that makes every evaluation functional $\hat{x}: X^* \to \mathbb{C}$, defined by $\hat{x}(g) = g(x)$ for a fixed $x \in X$, continuous.
    A sequence converges in a topology if and only if it converges with respect to every continuous linear functional. Therefore, for every $x \in X$, the continuous functional $\hat{x}$ satisfies:
    \begin{align}
        \lim_{n \to \infty} \hat{x}(f_n) = \hat{x}(f) \implies \lim_{n \to \infty} f_n(x) = f(x).
    \end{align}

    ($\Leftarrow$) 

    Assume $\lim_{n \to \infty} f_{n}(x) = f(x)$ for every $x \in X$.
    Let $V$ be an arbitrary basic open neighborhood of $f$ in the $\sigma(X^*, X)$ topology. 
    \begin{align}
        V = \{ g \in X^* : |g(x_i) - f(x_i)| < \epsilon_i \text{ for } i=1, \dots, k \}
    \end{align}
    for some finite set of vectors $\{x_1, \dots, x_k\} \subset X$ and $\epsilon_1, \dots, \epsilon_k > 0$.
    Since $\lim_{n \to \infty} f_{n}(x_i) = f(x_i)$ for each $i$, for every $\epsilon_i > 0$, there exists $N_i$ such that for all $n > N_i$, $|f_n(x_i) - f(x_i)| < \epsilon_i$.
    Let $N = \max \{N_1, \dots, N_k\}$. Then for all $n > N$, $f_n$ satisfies all $k$ conditions, meaning $f_n \in V$. Thus, $f_n$ converges to $f$ in the $\sigma(X^*, X)$ topology: $f_{n} \xrightarrow{*} f$.
\end{proof}

\subsection*{b)}

Statement: If $f_{n} \stackrel{w}{\rightarrow}  f$, then $f_{n} \xrightarrow{*} f$.

\begin{proof}
    Let \( J : X \to X^{**} \) be the canonical embedding defined by
    \begin{align}
        (Jx)(\varphi) = \varphi(x), \quad \forall x \in X, \, \varphi \in X^*.
    \end{align}
    Weak convergence \( f_n \xrightarrow{w} f \) in \( X^* \) means
    \begin{align}
        f_n(\Phi) \to f(\Phi), \quad \forall \Phi \in X^{**},
    \end{align}
    i.e., convergence with respect to the weak topology \( \sigma(X^*, X^{**}) \).

    Since every \( x \in X \) gives an element \( Jx \in X^{**} \), we have, for each \( x \in X \),
    \begin{align}
        f_n(x) = f_n(Jx) \longrightarrow f(Jx) = f(x).
    \end{align}
    Thus \( f_n(x) \to f(x) \) for every \( x \in X \), which by part (a) meas
    \begin{align}
        f_n \xrightarrow{*} f.
    \end{align}

\end{proof}


\subsection*{c)}

Statement: If $f_{n} \xrightarrow{*} f$, then $(f_{n})$ is bounded and $||f||_{X^{*}}\le \liminf_{n \to \infty}||f_{n}||_{X^{*}}$.

\begin{proof}
    
    Since $f_{n} \xrightarrow{*} f$, part (a) shows that $\lim_{n \to \infty} f_n(x) = f(x)$ for every $x \in X$.
    Since every convergent sequence of complex numbers is bounded, for every fixed $x \in X$, the scalar sequence $(f_n(x))$ is bounded:
    \begin{align}
        \sup_{n \in \mathbb{N}} |f_n(x)| < \infty
    \end{align}
    The sequence of functionals $(f_n)$ is therefore pointwise bounded on $X$. Since $X$ is a Banach space, with the Uniform Boundedness Principle applies, we have:
    \begin{align}
        \sup_{n \in \mathbb{N}} \norm{f_n}_{X^{*}} = M < \infty
    \end{align}
    Thus, $(f_n)$ is bounded in $X^*$.

    By the definition of the dual norm, for any $x \in X$ with $\norm{x}_X \le 1$:
    \begin{align}
        |f_n(x)| 
        \le \norm{f_n}_{X^{*}} \norm{x}_X 
        \le \norm{f_n}_{X^{*}}
    \end{align}

    Taking the limit inferior of both sides of the inequality:
    \begin{align}
        \liminf_{n \to \infty} |f_n(x)| \le \liminf_{n \to \infty} \norm{f_n}_{X^{*}}
    \end{align}
    Since $f_n(x) \to f(x)$, $\liminf |f_n(x)| = \lim |f_n(x)| = |f(x)|$. Thus:
    \begin{align}
        &|f(x)| \le \liminf_{n \to \infty} \norm{f_n}_{X^{*}}
        \\
        &\therefore \sup_{\norm{x}_X \le 1} |f(x)| \le \sup_{\norm{x}_X \le 1} \left( \liminf_{n \to \infty} \norm{f_n}_{X^{*}} \right)
        \\
        &\therefore \norm{f}_{X^{*}} \le \liminf_{n \to \infty} \norm{f_n}_{X^{*}}
    \end{align}
\end{proof}

\subsection*{d)}

Statement: If $f_{n} \xrightarrow{*} f$ and $x_{n} \to x$, then $\langle f_{n}, x_{n} \rangle \rightarrow \langle f, x \rangle$.

\begin{proof}
    We want to show $\lim_{n \to \infty} |f_n(x_n) - f(x)| = 0$. We use the triangle inequality trick:
    \begin{align}
    |f_n(x_n) - f(x)| 
    = |f_n(x_n) - f_n(x) + f_n(x) - f(x)|
    \\
    \le |f_n(x_n) - f_n(x)| + |f_n(x) - f(x)|
    \end{align}

    For the second term of (15), since $f_{n} \xrightarrow{*} f$ and $x$ is a fixed vector, by part (a):
    \begin{align}
        \lim_{n \to \infty} |f_n(x) - f(x)| = 0
    \end{align}

    For the first term of (15)
    \begin{align}
        |f_n(x_n) - f_n(x)| 
        = |f_n(x_n - x)|
        \\
        \le \norm{f_n}_{X^{*}} \|x_n - x\|_X
    \end{align}
    
    From part (c), we know that if $f_{n} \xrightarrow{*} f$, the sequence of norms is uniformly bounded:
    \begin{align}
        &\sup_{n \in \mathbb{N}} \norm{f_n}_{X^{*}} = M < \infty
    \end{align}
    Therefore,
    \begin{align}
        \lim_{n \to \infty} |f_n(x_n) - f_n(x)| 
        &\le \lim_{n \to \infty} (\norm{f_n}_{X^{*}} \|x_n - x\|_X) 
        \\
        &\le M \cdot \lim_{n \to \infty} \|x_n - x\|_X
    \end{align}
    
    Since $x_{n} \to x$, $\lim_{n \to \infty} \|x_n - x\|_X = 0$.
    \begin{align}
    0 \le \lim_{n \to \infty} |f_n(x_n) - f_n(x)| \le M \cdot 0 = 0
    \end{align}

    Thus, both terms in (15) converge to 0. Then,
    \begin{align}
        \lim_{n \to \infty} |f_n(x_n) - f(x)| = 0
    \end{align}
    , which means
    $
    \langle f_{n}, x_{n} \rangle \rightarrow \langle f, x \rangle
    $.
\end{proof}


\hrule
\vspace{0.5em}

% ---------------------------
% Problem 2
% ---------------------------

\begin{problem}
    Let $(X,||\cdot||_{X})$ be a Banach space and $X^{*}$ its dual. Given sequences $(x_{n})_{n} \subset X$ and $(f_{n})_{n} \subset X^{*}$. If $f_{n} \xrightarrow{*} f$ and $x_{n}\xrightarrow{w} x$, then $\langle f_{n},x_{n}\rangle\rightarrow\langle f,x\rangle.$
\end{problem}
% ------------- Solution

\vspace{0.5em}

The statement is false.

We give a counterexample in $l^2$.

Consider the Banach space $X = l^2(\mathbb{N})$, which is a Hilbert space and therefore reflexive, so $X^* = l^2$.
The pairing is given by $\langle f, x \rangle = \sum_{k=1}^\infty f_k x_k$.

Let $(e_n)$ be the standard basis vectors, where $e_n$ has a $1$ in the $n$-th position and $0$ elsewhere.

Let $x_n = e_n$.
With $x_{n}\xrightarrow{w} x$: For any $f = (f_k) \in l^2$, we have $\langle f, x_n \rangle = f_n$. Since $f \in l^2$, $f_n \rightarrow 0$ as $n \rightarrow \infty$.
Thus, $x_n \to 0$, so $x = 0$.

Next, we analysis the function sequence.
Let $f_n = e_n$. Note that $f_n \in X^*$.
With $f_{n} \xrightarrow{*} f$, for any $x = (x_k) \in l^2 = X$, we have $\langle f_n, x \rangle = x_n$. Since $x \in l^2$, $x_n \rightarrow 0$ as $n \rightarrow \infty$.
Thus, $f_n \stackrel{w^*}{\rightarrow} 0$, so $f = 0$.


We check the limit of the sequence of pairings $\langle f_{n}, x_{n} \rangle$:
\begin{align}
    \lim_{n\to\infty} \langle f_{n}, x_{n} \rangle = \lim_{n\to\infty} \langle e_n, e_n \rangle = \lim_{n\to\infty} (1) = 1.
\end{align}
The claimed limit is the pairing of the weak limits:
\begin{align}
    \langle f, x \rangle = \langle 0, 0 \rangle = 0.
    \\
    \therefore
    \langle f_{n}, x_{n} \rangle \not\rightarrow \langle f, x \rangle.
\end{align}


\hrule
\vspace{0.5em}

% ---------------------------
% Problem 3
% ---------------------------
\begin{problem}
\end{problem}
% ------------- Solution
\begin{proof}

% The key result in functional analysis we use is Kakutani's Theorem (a consequence of the Banach-Alaoglu Theorem): a Banach space $X$ is reflexive if and only if its closed unit ball $B_X = \{x \in X : \norm{x} \le 1\}$ is compact in the weak topology $\sigma(X, X^*)$.

The map $T: E \to F$ is a linear surjective isometry. Thus, we have: $\|T(x)\|_F = \norm{x}_E$ for all $x \in E$.

Therefore, $T$ is injective (since $\|T(x)\|_F = 0 \iff \norm{x}_E = 0 \iff x=0$), and then $T$ is a bijection. 

Its inverse $T^{-1}: F \to E$ also exists and is a linear isometry. Hense, both $T$ and $T^{-1}$ are bounded operators with operator norms $\|T\| = 1$ and $\|T^{-1}\| = 1$.

\subsection*{$E$ is Reflexive $\implies$ $F$ is Reflexive}

Since $T$ is an isometry, $T$ maps the closed unit ball of $E$ onto the closed unit ball of $F$:
$T(B_E) = B_F$

Since $E$ is reflexive, its closed unit ball $B_E$ is $\sigma(E, E^*)$-compact.

Since $T$ is norm-continuous, it is weakly continuous, as a linear map between normed spaces is norm-continuous if and only if it is weakly continuous.

Since $B_E$ is $\sigma(E, E^*)$-compact and $T$ is weakly continuous, the image $T(B_E) = B_F$ must be $\sigma(F, F^*)$-compact, as the continuous image of a compact set is compact. 

Thus, $B_F$ is weakly compact in $F$. Therefore, $F$ is reflexive.

\subsection*{$F$ is Reflexive $\implies$ $E$ is Reflexive}

As we have shown that $T^{-1}$ is also a linear isometry. With the same process of the previous part, $F$ is reflexive $\implies$ $E$ is reflexive.

\end{proof}


\hrule
\vspace{0.5em}

\end{document}