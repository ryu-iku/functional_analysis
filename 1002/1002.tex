\documentclass{article}
\usepackage{amsmath, amssymb, amsthm}
\usepackage{geometry}
\geometry{a4paper, margin=1in}

\DeclareMathOperator{\esssup}{ess\,sup}
\newcommand{\norm}[1]{\left\|#1\right\|}
\newcommand{\abs}[1]{\left|#1\right|}
\newcommand{\set}[1]{\left\{#1\right\}}
\newcommand{\R}{\mathbb{R}}
\newcommand{\C}{\mathbb{C}}
\newcommand{\N}{\mathbb{N}}
\newcommand{\ellp}{\ell^p}
\newcommand{\ellq}{\ell^q}
\newcommand{\ip}[2]{\left\langle #1, #2 \right\rangle}

\newtheorem{problem}{Problem}

\begin{document}

\begin{center}
    \textbf{\Large Functional Analysis - Homework 4}
\end{center}

\hrule
\vspace{0.5em}

% ---------------------------
% Problem 1
% ---------------------------

\begin{problem}

\end{problem}

\begin{proof}

    Let $X = [1,2] \subset \mathbb{R}$, a complete metric space.

    Let $\epsilon > 0$, $m \in \mathbb{N}$, and 
    $$
    A_m = \{ x \in X : \forall n \geq m, \abs{f(nx)} \leq \epsilon \}
    $$

    Since $f$ is continuous, $A$ is closed.

    $\forall x \in X$, since $\lim_{n \to \infty}(nx) = 0$
    $$
    \exists n > N, \abs{f(nx)} \leq \epsilon
    $$
    , meaning $x$ should be contained in one of $A_m$s.

    Thus,
    $$
    X = \bigcup_m A_m
    $$

    With Baire’s category Theorem, at least one $A_m$s contains an open ball $B(b_0, r)$, where $b_0 \in X, r > 0$.

    Therefore,
    \begin{gather}
    \forall n > m, b < r, \abs{f(n(b_0 + b))} \leq \epsilon 
    \\
    \because n\frac{b_0 + r}{b_0 + b} > m
    \therefore \abs{f(n(b_0 + r))} \leq \epsilon
    \end{gather}

    
    For $x \in (0, \infty)$, with $n > \frac{m}{x}(b_0 + r)$, we have
    $$
    \abs{f(nx)} = \abs{f(\frac{m}{x}(b_0 + r) x)}
    = \abs{f(\frac{m}{x}(b_0 + r) x)}  \leq \epsilon .
    $$
    , which means $\lim_{t \to \infty} f(t) = 0$.

    For $x = 0$, from $\lim_{n \to \infty} f(n 0) = 0$, we know $f(0) = 0$.

    In a whole, For $x \in [0, \infty)$, $\lim_{t \to \infty} f(t) = 0$.

    
\end{proof}

\hrule
\vspace{0.5em}

% ---------------------------
% Problem 2
% ---------------------------

\begin{problem}

\end{problem}

\begin{proof}
\subsection*{1. Linearity of $c_0(X)$}
Let $x = (x_n), y = (y_n) \in c_0(X)$ and $\lambda \in \mathbb{K}$.
\begin{itemize}
    \item Addition: $x+y = (x_n + y_n)$. By the triangle inequality on $X$:
    \begin{gather}
        0 
        \leq \lim_{n\to\infty}\norm{x_n + y_n} 
        \leq \lim_{n\to\infty}(\norm{x_n} + \norm{y_n})
        = \lim_{n\to\infty}\norm{x_n} + \lim_{n\to\infty}\norm{y_n} = 0
        \\
        \therefore \lim_{n\to\infty} \norm{x_n + y_n} = 0
    \end{gather}
    
    Thus, $x+y \in c_0(X)$.

    \item Scalar Multiplication: $\lambda x = (\lambda x_n)$. By the properties of a norm:
    $$
    \lim_{n\to\infty} \norm{\lambda x_n} = \lim_{n\to\infty} \abs{\lambda} \norm{x_n} = \abs{\lambda} \lim_{n\to\infty} \norm{x_n} = \abs{\lambda} \cdot 0 = 0
    $$
    Thus $\lambda x \in c_0(X)$.
\end{itemize}
Hence, $c_0(X)$ is a linear space.

\subsection*{2. $X$ is Banach $\implies$ completeness of $c_0(X)$}
    Assume $X$ is a Banach space. We show that $c_0(X)$ is complete with respect to the norm $\norm{x} = \sup_{n\in\mathbb{N}} \norm{x_n}_X$.
    
    Let $(x^{(k)})_{k\in\mathbb{N}}$ be a Cauchy sequence in $c_0(X)$, where $x^{(k)} = (x_1^{(k)}, x_2^{(k)}, \ldots, x_n^{(k)}, \ldots)$.
    
    Since $(x^{(k)})_{k\in\mathbb{N}}$ is Cauchy, for every $\epsilon > 0$, there exists an integer $K$ such that for all $k, j \geq K$:
    \begin{gather}
    \norm{x^{(k)} - x^{(j)}} = \sup_{n\in\mathbb{N}} \norm{x_n^{(k)} - x_n^{(j)}}_X < \epsilon \label{eq:cauchy_sup}
    \end{gather}
    
    \begin{itemize}
        \item \textbf{Step 1: Determine the limit element $x$}.
        For a fixed $n \in \mathbb{N}$, the sup-norm condition $(\ref{eq:cauchy_sup})$ implies that the component sequence $(x_n^{(k)})_{k\in\mathbb{N}}$ is Cauchy in $X$, since:
        \begin{gather}
        \norm{x_n^{(k)} - x_n^{(j)}}_X \leq \sup_{m\in\mathbb{N}} \norm{x_m^{(k)} - x_m^{(j)}}_X < \epsilon \quad \text{for all } k, j \geq K
        \end{gather}
        Since $X$ is a Banach space, it is complete. Therefore, for each $n$, the sequence $(x_n^{(k)})_{k\in\mathbb{N}}$ converges to some limit $x_n \in X$:
        \begin{gather}
        x_n = \lim_{k\to\infty} x_n^{(k)}
        \end{gather}
        We define the candidate limit element as $x = (x_n)_{n\in\mathbb{N}}$.
        
        \item \textbf{Step 2: Show $x \in c_0(X)$}.
        We must show $\lim_{n\to\infty} \norm{x_n}_X = 0$.
        Fix $k \geq K$. Taking the limit as $j \to \infty$ in the inequality $\norm{x_n^{(k)} - x_n^{(j)}}_X \leq \epsilon$, which is a consequence of the Cauchy condition $(\ref{eq:cauchy_sup})$, we get:
        \begin{gather}
        \norm{x_n^{(k)} - x_n}_X \leq \epsilon \quad \text{for all } n \in \mathbb{N}
        \end{gather}
        Now, use the triangle inequality on $X$:
        \begin{gather}
        \norm{x_n}_X \leq \norm{x_n - x_n^{(k)}}_X + \norm{x_n^{(k)}}_X \leq \epsilon + \norm{x_n^{(k)}}_X
        \end{gather}
        Since $x^{(k)} \in c_0(X)$, we know $\lim_{n\to\infty} \norm{x_n^{(k)}}_X = 0$.
        Thus, for this fixed $k$, there exists an integer $N$ such that for all $n \geq N$:
        \begin{gather}
        \norm{x_n^{(k)}}_X < \epsilon
        \end{gather}
        
        Combining the two preceding inequalities (for $\norm{x_n}_X$ and $\norm{x_n^{(k)}}_X$) for $n \geq N$:
        \begin{gather}
        \norm{x_n}_X < \epsilon + \epsilon = 2\epsilon
        \end{gather}
        Since $\epsilon > 0$ was arbitrary, this proves $\lim_{n\to\infty} \norm{x_n}_X = 0$, so $x \in c_0(X)$.
        
        \item \textbf{Step 3: Show $x^{(k)} \to x$ in $c_0(X)$}.
        The inequality $\norm{x_n^{(k)} - x_n}_X \leq \epsilon$ (derived in Step 2 for $k \geq K$) holds for all $n \in \mathbb{N}$.
        Taking the supremum over $n$:
        \begin{gather}
        \norm{x^{(k)} - x} = \sup_{n\in\mathbb{N}} \norm{x_n^{(k)} - x_n}_X \leq \epsilon \quad \text{for all } k \geq K
        \end{gather}
        Therefore, $x^{(k)}$ converges to $x$ in $c_0(X)$.
    \end{itemize}
    Since every Cauchy sequence in $c_0(X)$ converges to an element in $c_0(X)$, the space $c_0(X)$ is a Banach space.

% 2. $X$ is Banach $\Longrightarrow$ completeness of $c_0(X)$ :

% Assume $X$ is a Banach space. Let $(x^{(k)})_{k\in\N}$ be a Cauchy sequence in $c_0(X)$ with respect to the $\sup$-norm, where $x^{(k)} = (x_1^{(k)}, x_2^{(k)}, \ldots)$, and let it converge to $x$.

% Since the sequence is Cauchy, we have
% $$
% \forall \epsilon > 0, \exists k > N \text{ s.t. } 
% \norm{x^{(k)} - x} < \epsilon, 
% $$

% \begin{gather}
% \therefore \epsilon > \norm{x^{(k)} - x}
% = \sup_{n\in\N} \norm{x_n^{(k)} - x_n}_{X}
% \end{gather}

% Thus, $\forall m \in \N$,
% \begin{gather}
% \norm{x_m^{(k)} - x_m}_X
% \leq \sup_{n\in\N} \norm{x_n^{(k)} - x_n}_{X}
% < \epsilon
% \end{gather}

% With trianglar inequality, we have
% \begin{gather}
% \norm{x_m}_{X} 
% \leq \norm{x_m - x_m^{(k)}}_{X} + \norm{x_m^{(k)}}_{X}
% \end{gather}

% From (6), we have
% \begin{gather}
% \norm{x_m^{(k)} - x_m}_X + \norm{x_m^{(k)}}_{X} 
% < \epsilon + \norm{x_m^{(k)}}_{X}
% \end{gather}

% Get (7) and (8) togeter, wa have
% \begin{gather}
% \norm{x_m}_{X}
% < \epsilon + \norm{x_m^{(k)}}_{X}
% \end{gather}

% Since $x^{(k)} \in c_0(X)$, $\lim_{n\to\infty} \norm{x_n^{(k)}}_{X} = 0$. 
% Therefore, there exists $N \in \N$ such that for all $n \geq N$, $\norm{x_n^{(k)}}_{X} < \epsilon$.

% Hence, for $l \geq N$ with (9)
% \begin{gather}
% \norm{x_l}_{X} 
% \leq \epsilon + \norm{x_l^{(k)}}_{X} 
% = 2\epsilon
% \end{gather}

% Since $\epsilon > 0$ was arbitrary, this proves 
% \begin{gather}
% \lim_{l\to\infty} \norm{x_l}_{X} = 0
% \end{gather}
% , so $x \in c_0(X)$.
% Since every Cauchy sequence in $c_0(X)$ converges to an element in $c_0(X)$, the space $c_0(X)$ is Banach.
\end{proof}

\hrule
\vspace{0.5em}

% ---------------------------
% Problem 3
% ---------------------------

\begin{problem}
\end{problem}

\begin{proof}

\textbf{(a) $\implies$ (b)}

We define a new space 
$$
Z = \{ (x_n)_{n \in \N} : \norm{x_n} \rightarrow 0 \}
$$
and norm $ \norm{(x_n)_{n \in \N}} = \sup_{n} \norm{x_n} $.

% Since $(x_n)_{n \in \N}$ converge to 0, 
% $$
% \forall \frac{\epsilon}{2} > 0, \exists m > N, \text{ .s.t. } \norm{z_m} \leq \norm{x_m} + \norm{x_{m+1}} \leq \frac{\epsilon}{2} + \frac{\epsilon}{2} = \epsilon
% $$
% Therefore, $(z_n)_{n \in \N}$ converge to 0 as well. 
According to the proof of Problem 2, $Z$ is Banach.

We define an operator $S_k: Z \rightarrow Y$
$$
S_k(z) = T_k(x_k - x_{k+1}).
$$
Therefore,
\begin{gather}
\norm{S_k(z)} = \norm{T_k(x_k - x_{k+1})} 
\\
\leq \norm{T_k} \norm{x_k - x_{k+1}} 
\\
\leq \norm{T_k} (\norm{x_k} + \norm{x_{k+1}})
\\
\leq \norm{T_k} \cdot 2 \sup_{n \in \N} \norm{x_n}
\end{gather}

Therefore, 
$$
\norm{S_k}  \leq 2 \norm{T_k} < \infty
$$
as $T_k$ is continuous.

Since $T_k(x_k) \rightarrow 0$ in norm, $S_k$ should be the same.
\begin{gather}
\lim_{k \to \infty} \norm{S_k(z)} = 0
\\
\therefore \sup_{k \in \N} \norm{S_k(z)} < \infty
\end{gather}

With the Uniform Boundedness Principle, we have
\begin{gather}
    \sup_{k \in \N} \norm{S_k} < \infty
\end{gather}


$
\text{For any } t \in X \text{ with }  \norm{t} \le 1 \text{, define the sequence } 
z' = (x_n)_{n \in \N}
$,  where $x_n = t$ if $n=k$, and $x_n = 0$ else.
Then,
$
\sup_{n} x_n = \norm{t} \le 1.
$
Applying the operator $S_k$,
\begin{gather}
    \norm{S_k(z')} = \norm{T_k(x_k - x_{k+1})} = \norm{T_k(x_k)} 
    \\
    \therefore \norm{S_k} \geq \sup_{\norm{t} \leq 1} \norm{T_k(t)} = \norm{T_k}
\end{gather}
Since $\sup_{k \in \N} \norm{S_k} < \infty$,
$$
\norm{T_k} < \infty
$$

\textbf{Proof: (b) $\implies$ (a)}
Assume (b) holds, i.e., $M = \sup_{n\in\N} \norm{T_n} < \infty$.
Assume $\sum_{n=1}^{\infty}x_{n}$ is a norm convergent series. Let $s$ be the sum. The sequence of partial sums $s_N = \sum_{n=1}^N x_n$ converges to $s$.
This implies that the terms of the series must converge to zero: $\lim_{n\to\infty} x_n = 0$ in the norm of $X$.

$$
\norm{T_n(x_n)}_Y \leq \norm{T_n} \cdot \norm{x_n}_X
$$
Since $\sup_{n\in\N} \norm{T_n} = M < \infty$, we have:
$$
0 \leq \norm{T_n(x_n)}_Y \leq M \cdot \norm{x_n}_X
$$
Since $\lim_{n\to\infty} \norm{x_n}_X = 0$ and $M$ is a finite constant, we have $\lim_{n\to\infty} M \cdot \norm{x_n}_X = 0$.
By the Squeeze Theorem, $\lim_{n\to\infty} \norm{T_n(x_n)}_Y = 0$. Thus, $T_n(x_n) \to 0$ in norm.

\end{proof}


\hrule
\vspace{0.5em}

% ---------------------------
% Problem 4
% ---------------------------

\begin{problem}
\end{problem}

\begin{proof}

\textbf{(a) $\implies$ (b):}
The set of $p$-absolutely norm convergent sequences $E = \ell_p(X)$ is a Banach space with the norm $\norm{(x_n)}_p = \left(\sum_{n=1}^{\infty} \norm{x_n}^p\right)^{1/p}$.

(a) states that for every sequence $(x_n) \in E$, the sum $\sum_{n=1}^{\infty}x_{n}^{*}(x_{n})$ converges. This allows us to define a linear functional $T: E \to \mathbb{K}$ by:
$$
T((x_n)) = \sum_{n=1}^{\infty} x_{n}^{*}(x_{n})
$$

Since $T$ is a well-defined linear map on the entire Banach space $E$, the Uniform Boundedness Principle implies that $T$ is bounded and in turn continuous. We have $T \in E^*$.

With Dual Space Isomorphism, the dual space of $\ell_p(X)$ is isometrically isomorphic to $\ell_q(X^*)$:

$$\ell_p(X)^* \cong \ell_q(X^*)$$

$(x_n^*)$ corresponds to the functional $T$, and the norm of $T$ in the dual space is precisely the $\ell_q$ norm of the sequence $(x_n^*)$ in the dual sequence space:
$$
\norm{T}_{E^*} = \left(\sum_{n=1}^{\infty} \norm{x_{n}^{*}}^{q}\right)^{1/q}
$$

Since $T$ is continuous, $\norm{T}_{E^*} < \infty$, which means $\sum_{n=1}^{\infty}\norm{x_{n}^{*}}^{q}<\infty$. Thus, (b) holds.


\textbf{(b) $\implies$ (a):}
Assume (b) holds: $(x_n^*) \in \ell_q(X^*)$, so $A = \left(\sum_{n=1}^{\infty} \norm{x_{n}^{*}}^{q}\right)^{1/q} < \infty$.

Assume the series $\sum_{n=1}^{\infty} x_n$ is $p$-absolutely norm convergent: 
$(x_n) \in \ell_p(X)$, so 
$$
B = \left(\sum_{n=1}^{\infty} \norm{x_{n}}^{p}\right)^{1/p} < \infty
$$

With the Holder's Inequality, we have:
\begin{gather}
    \sum_{n=1}^{\infty} \abs{x_{n}^{*}(x_{n})} \leq 
    \sum_{n=1}^{\infty} \norm{x_{n}^{*}} \cdot \norm{x_{n}} 
    \\
    \leq \left(\sum_{n=1}^{\infty} \norm{x_{n}^{*}}^{q}\right)^{1/q} \left(\sum_{n=1}^{\infty} \norm{x_{n}}^{p}\right)^{1/p} = A \cdot B < \infty
\end{gather}


\vspace{0.5em}
\textbf{(c) $\iff$ (d)}

Let $X=\mathbb{K}$. Then the dual space $X^*$ is also $\mathbb{K}$.

A continuous linear functional $x_n^*: \mathbb{K} \to \mathbb{K}$ is simply multiplication by a scalar $x_n \in \mathbb{K}$. The norm of this functional is $||x_n^*|| = |x_n|$. The input vector $x_n$ from statement (a) is now a scalar $y_n$.

Thus, statement (a) becomes:
Given a series $\sum_{n=1}^{\infty} y_{n}$ such that $\sum_{n=1}^{\infty}|y_{n}|^{p}<\infty$ (i.e., $(y_n) \in \ell_p$) one has that the series $\sum_{n=1}^{\infty}x_{n}y_{n}$ converges.
This is precisely statement (d). So we have (d) $\implies$ (a).


And statement (b) becomes: 
The series $\sum_{n=1}^{\infty}x_{n}^{*}$ is $q$-absolutely norm convergent i.e., $\sum_{n=1}^{\infty}\norm{x_{n}^{*}}^{q}<\infty$.
Since $\norm{x_{n}^{*}} = |x_n|$, this simplifies to $\sum_{n=1}^{\infty}|x_{n}|^{q}<\infty$, which means $(x_n) \in \ell_q$.
This is precisely statement (c). So we have (d) $\implies$ (a).

Since (a) is equivalent to (b), (d) is equivalent to (c).
\end{proof}

\end{document}