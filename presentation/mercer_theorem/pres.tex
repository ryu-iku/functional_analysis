\documentclass[11pt, a4paper]{article}

% Use standard LaTeX packages compatible with pdflatex/latex engine
\usepackage[utf8]{inputenc}
\usepackage[T1]{fontenc}
\usepackage[english]{babel}
\usepackage{amsmath, amssymb, amsthm}
\usepackage{geometry}
\usepackage{upgreek} % For upright Greek letters if needed, though not used here

% Set geometry
\geometry{a4paper, top=2.5cm, bottom=2.5cm, left=2cm, right=2cm}

% Define theorem and definition environments
\newtheorem{definition}{Definition}[section]
\newtheorem{theorem}{Theorem}[section]
\newtheorem{corollary}{Corollary}[section]
\newtheorem{lemma}{Lemma}[section]
\newtheorem{remark}{Remark}[section]

% Remove extra space from theorem definitions
\makeatletter
\def\thm@space@setup{%
  \thm@preskip=\topsep \thm@postskip=\topsep
  \itemsep \z@ \topsep \z@ \partopsep \z@ \parsep \z@
}
\makeatother

\begin{document}

\section{Mercer’s Theorem and Feature Map} \label{section:mercers_theorem_and_feature_map}

\subsection{Mercer’s Theorem} \label{section:mercers_theorem}

\begin{definition}[Definite Kernel] \label{definition:definite_kernel}
The function $k : [a, b] \times [a, b] \rightarrow \mathbb{R}$ is a definite kernel where the following double integral:
\begin{align}
J(f) = \int_{a}^{b} \int_{a}^{b} k(x, y) f(x) f(y) dx dy, \label{equation:definite_kernel_integral}
\end{align}
satisfies $J(f) > 0$ for all $f(x) \neq 0$.
\end{definition}

Mercer improved over Hilbert’s work to propose his theorem, the Mercer’s theorem, introduced in the following.

\begin{theorem}[Mercer’s Theorem] \label{theorem:mercers_theorem_main}
Suppose $k : [a, b] \times [a, b] \rightarrow \mathbb{R}$ is a continuous symmetric positive semi-definite kernel which is bounded:
\begin{align}
\sup_{x, y} k(x, y) < \infty. \label{equation:mercer_bounded}
\end{align}
Assume the operator $T_k$ 
% takes a function $f(x)$ as its argument and outputs a new function as:
\begin{align}
T_k f(x) := \int_{a}^{b} k(x, y) f(y) dy, \label{equation:mercer_operator}
\end{align}
% which is a Fredholm integral equation. The operator $T_k$ is called the Hilbert–Schmidt integral operator. 
% This output function 
is positive semi-definite:
\begin{align}
\iint k(x, y) f(y) dx dy \geq 0. \label{equation:mercer_positive_semidefinite_integral}
\end{align}

1) Then, there is a set of orthonormal bases $\{ \psi_i(.) \}_{i=1}^\infty$ of $L^2(a, b)$ consisting of eigenfunctions of $T_K$ such that the corresponding sequence of eigenvalues $\{ \lambda_i \}_{i=1}^\infty$ are non-negative:
\begin{align}
\int k(x, y) \psi_i(y) dy = \lambda_i \psi_i(x). \label{equation:mercer_eigenfunction_decomposition}
\end{align}

2) The eigenfunctions corresponding to the non-zero eigenvalues are continuous on $[a, b]$ and $k$ can be represented as:
\begin{align}
k(x, y) = \sum_{i=1}^\infty \lambda_i \psi_i(x) \psi_i(y), \label{equation:mercer_kernel_decomposition}
\end{align}
where the convergence is absolute and uniform.
\end{theorem}

\begin{proof}
The proof relies on establishing that the operator $T_k$ is symmetric and compact, which allows the application of the Spectral Theorem for compact self-adjoint operators.

\textbf{Step 1: The Operator $T_k$ is Symmetric and Compact.}
According to the theorem's assumptions, the Hilbert-Schmidt integral operator $T_k$ is a symmetric operator on the Hilbert space $L^2(a, b)$ and is also a compact operator.

\begin{itemize}
    \item \textbf{Symmetry:} The symmetry of $T_k$ follows from the symmetry of the kernel $k(x,y)$, i.e., $k(x,y) = k(y,x)$. For any $f, g \in L^2(a, b)$, we must show $\langle T_k f, g \rangle_{L^2} = \langle f, T_k g \rangle_{L^2}$. Expanding the left side:
    \begin{align*}
    \langle T_k f, g \rangle_{L^2} &= \int_{a}^{b} (T_k f(x)) g(x) dx \\
    &= \int_{a}^{b} \left( \int_{a}^{b} k(x, y) f(y) dy \right) g(x) dx
    \end{align*}
    Since $k(x,y)$ is continuous and bounded, the integrand is absolutely integrable, allowing us to interchange the order of integration using \textbf{Fubini's Theorem}:
    \begin{align*}
    \langle T_k f, g \rangle_{L^2} &= \int_{a}^{b} \int_{a}^{b} k(x, y) f(y) g(x) dx dy \\
    &= \int_{a}^{b} f(y) \left( \int_{a}^{b} k(y, x) g(x) dx \right) dy \quad (\text{using } k(x,y) = k(y,x))\\
    &= \int_{a}^{b} f(y) (T_k g(y)) dy = \langle f, T_k g \rangle_{L^2}.
    \end{align*}
    Thus, $T_k$ is symmetric (self-adjoint).

    \item \textbf{Compactness:} The compactness of $T_k$ is guaranteed because $k(x,y)$ is continuous and bounded on the compact domain $[a,b] \times [a,b]$. The operator $T_k$ maps bounded sets in $L^2(a,b)$ to relatively compact sets in $C([a,b])$. This relies on the \textbf{Arzel\`a-Ascoli Theorem}, which requires that the image set $\mathcal{F} = \{T_k f \mid \|f\|_{L^2} \le 1\}$ is uniformly bounded and equicontinuous. Both conditions are satisfied due to the uniform continuity and boundedness of $k(x,y)$.
\end{itemize}

\textbf{Step 2: Application of the Spectral Theorem and Eigenvalue Decomposition.}
Since $T_k$ is a symmetric (self-adjoint) and compact operator on the Hilbert space $L^2(a, b)$, the \textbf{Spectral Theorem for compact self-adjoint operators} applies. This theorem guarantees:
\begin{enumerate}
    \item The existence of an orthonormal basis $\{ \psi_i(.) \}_{i=1}^\infty$ for $L^2(a, b)$ consisting of the eigenfunctions of $T_k$.
    \item That the operator satisfies the eigenvalue decomposition:
    \begin{align}
    T_k \psi_i(x) = \lambda_i \psi_i(x), \label{equation:spectral_decomposition_operator}
    \end{align}
    where $\{ \lambda_i \}_{i=1}^\infty$ are the corresponding real eigenvalues.
\end{enumerate}
Substituting the definition of $T_k$ from Eq. \eqref{equation:mercer_operator} into Eq. \eqref{equation:spectral_decomposition_operator} yields the eigenfunction decomposition:
\begin{align}
\int k(x, y) \psi_i(y) dy \overset{\eqref{equation:mercer_operator}}{=} T_k \psi_i(x) \overset{\eqref{equation:spectral_decomposition_operator}}{=} \lambda_i \psi_i(x). \label{equation:eigenfunction_decomposition_operator}
\end{align}
Furthermore, since $k(x,y)$ is positive semi-definite (Eq. \eqref{equation:mercer_positive_semidefinite_integral}), it guarantees that all eigenvalues $\lambda_i$ must be non-negative ($\lambda_i \geq 0$).

\textbf{Step 3: Derivation of the Kernel Expansion (Mercer's Series).}
The functions $\{ \psi_i(.) \}_{i=1}^\infty$ form a \textbf{complete orthonormal basis} for $L^2(a, b)$ due to the Spectral Theorem. This completeness means that any function in $L^2(a, b)$, including $g_x(y) = k(x,y)$ (for a fixed $x$), can be perfectly represented by the basis. This is formalized by \textbf{Parseval's Identity} (the equality case of Bessel's Inequality).
\begin{enumerate}
    \item We expand $k(x, y)$ (as a function of $y$) in the orthonormal basis $\{ \psi_i(y) \}$:
    \begin{align}
    k(x, y) = \sum_{i=1}^\infty \langle k(x, \cdot), \psi_i \rangle_{L^2} \psi_i(y) \nonumber 
    \end{align}
    \item The coefficients are the inner products, which by the eigenvalue equation (Eq. \eqref{equation:eigenfunction_decomposition_operator}) are:
    \begin{align}
    \langle k(x, \cdot), \psi_i \rangle_{L^2} = \int_{a}^{b} k(x, y) \psi_i(y) dy = \lambda_i \psi_i(x). \label{equation:eigenvalue_substituion}
    \end{align}
    \item Substituting the coefficients gives Mercer's Series:
    \begin{align}
    k(x, y) = \sum_{i=1}^\infty (\lambda_i \psi_i(x)) \psi_i(y) = \sum_{i=1}^\infty \lambda_i \psi_i(x) \psi_i(y), \label{equation:mercer_kernel_decomposition_reproved}
    \end{align}
    which is Eq. \eqref{equation:mercer_kernel_decomposition}.
\end{enumerate}

\textbf{Step 4: Proving Absolute and Uniform Convergence.}
The final step is crucial as it elevates the convergence from $L^2$ (guaranteed by the Spectral Theorem) to the stronger \textbf{absolute and uniform convergence} required by the theorem.
\begin{itemize}
    \item \textbf{Meaning of Convergence:}
    \begin{enumerate}
        \item \textbf{Absolute Convergence:} The series $\sum_{i=1}^\infty |\lambda_i \psi_i(x) \psi_i(y)|$ must converge.
        \item \textbf{Uniform Convergence:} The rate of convergence of the partial sums $S_n(x,y) = \sum_{i=1}^n \lambda_i \psi_i(x) \psi_i(y)$ to $k(x,y)$ must be independent of the location $(x,y)$ in the domain. Uniform convergence is required to ensure that the sum $k(x,y)$ inherits the \textbf{continuity} property from its continuous terms (eigenfunctions $\psi_i(x)$).
    \end{enumerate}

    \item \textbf{The Truncated Kernel $r_n(x,y)$:} To prove this strong form of convergence, we must analyze the remainder of the series. We define the truncated kernel $r_n$ (with parameter $n$) as the remainder:
    \begin{align}
    r_n(x, y) &:= k(x, y) - \sum_{i=1}^n \lambda_i \psi_i(x) \psi_i(y) = \sum_{i=n+1}^\infty \lambda_i \psi_i(x) \psi_i(y). \label{equation:truncated_kernel}
    \end{align}
    The definition of the truncated kernel is essential because it allows us to utilize the positive semi-definite property of $k(x,y)$. Since $k(x,y)$ is positive semi-definite and each term $\lambda_i \psi_i(x) \psi_i(y)$ (with $\lambda_i \ge 0$) is also a rank-1 positive semi-definite kernel, their difference $r_n(x,y)$ is \textbf{also a positive semi-definite kernel}.

    \item \textbf{Establishing the Bound:} The positive semi-definite property implies that $r_n(x, x) \geq 0$ for every $x \in [a, b]$, which leads to:
    \begin{equation*} 
    r_n(x, x) = k(x, x) - \sum_{i=1}^n \lambda_i \psi_i(x)^2 \geq 0 
    \end{equation*}
    which establishes the critical upper bound for the partial sums:
    \begin{equation} 
    \sum_{i=1}^n \lambda_i \psi_i(x)^2 \leq k(x, x) \leq \sup_{x \in [a, b]} k(x, x). \label{equation:truncated_kernel_bound}
    \end{equation}
    Since $k(x,x)$ is bounded (Eq. \eqref{equation:mercer_bounded}), the partial sums for the series $\sum_{i=1}^\infty \lambda_i \psi_i(x)^2$ are uniformly bounded, ensuring its convergence.

    \item \textbf{Conclusion of Convergence and Uniform Bound:}
    The uniform boundedness of the partial sums for the diagonal terms, $\sum_{i=1}^n \lambda_i \psi_i(x)^2$ (Eq. \eqref{equation:truncated_kernel_bound}), is the key. Since the eigenvalues $\lambda_i$ are non-negative, the series $\sum_{i=1}^\infty \lambda_i \psi_i(x)^2$ converges pointwise and is bounded by $\sup_{x} k(x,x)$.
    
    We now apply the \textbf{Cauchy-Schwarz inequality for series} to bound the magnitude of the remainder term $r_n(x,y)$ (the sum from $i=n+1$ to $\infty$):
    \begin{align*}
    |r_n(x, y)| &= \left| \sum_{i=n+1}^\infty \lambda_i \psi_i(x) \psi_i(y) \right| \\
    &\leq \sum_{i=n+1}^\infty \lambda_i |\psi_i(x) \psi_i(y)| \quad (\text{Since } \lambda_i \ge 0, \text{ the absolute value moves inside}) \\
    &\leq \sqrt{\left( \sum_{i=n+1}^\infty \lambda_i \psi_i(x)^2 \right) \left( \sum_{i=n+1}^\infty \lambda_i \psi_i(y)^2 \right)}
    \end{align*}
    Let $R_n(x) = \sum_{i=n+1}^\infty \lambda_i \psi_i(x)^2$ be the remainder of the convergent series for $k(x,x)$. Since $k(x,y)$ is continuous on the compact domain, the convergence of the diagonal series $k(x,x) = \sum_{i=1}^\infty \lambda_i \psi_i(x)^2$ is itself uniform. This implies that $R_n(x) \to 0$ uniformly for all $x \in [a,b]$ as $n \to \infty$.

    Since both $R_n(x) \to 0$ and $R_n(y) \to 0$ uniformly, their product also goes to zero uniformly:
    $$ |r_n(x, y)| \leq \sqrt{R_n(x) R_n(y)} \to 0 \quad \text{uniformly as } n \to \infty. $$
    Because the remainder term $r_n(x,y)$ converges uniformly to zero, the original series $\sum_{i=1}^\infty \lambda_i \psi_i(x) \psi_i(y)$ converges \textbf{uniformly}. Furthermore, the boundedness of the series of absolute values (established via the Cauchy-Schwarz bound) proves \textbf{absolute convergence}.
    
    This strong form of convergence (absolute and uniform) is necessary to ensure the resulting function $k(x,y)$ is continuous, validating the series representation of the kernel, and concluding the proof.
\end{itemize}
\end{proof}

% \subsection{Feature Map and Pulling Function} \label{section:feature_map}
% Let $\mathcal{X} := \{x_i\}_{i=1}^n$ be the set of data in the input space (note that the input space is the original space of data). The $t$-dimensional (perhaps infinite dimensional) feature space (or Hilbert space) is denoted by $\mathcal{H}$.

% \begin{definition}[Feature Map or Pulling Function] \label{definition:feature_map}
% We define the mapping:
% \begin{align}
% \phi : \mathcal{X} \rightarrow \mathcal{H}, \label{equation:feature_map_definition}
% \end{align}
% to transform data from the input space to the feature space, i.e. Hilbert space. In other words, this mapping pulls data to the feature space:
% \begin{align}
% x \mapsto \phi(x). \label{equation:pulling_function}
% \end{align}
% The function $\phi(x)$ is called the feature map or pulling function. The feature map is a (possibly infinite-dimensional) vector whose elements are:
% \begin{align}
% \phi(x) = [\phi_1(x), \phi_2(x), \dots]^\top := [\sqrt{\lambda_1} \psi_1(x), \sqrt{\lambda_2} \psi_2(x), \dots]^\top, \label{equation:feature_map_elements}
% \end{align}
% where $\{ \psi_i \}$ and $\{ \lambda_i \}$ are eigenfunctions and eigenvalues of the kernel operator (see Eq. \eqref{equation:mercer_eigenfunction_decomposition}). Note that eigenfunctions will be explained more in Section \ref{section:eigenfunctions}.
% \end{definition}

% Let $t$ denote the dimensionality of $\phi(x)$. The feature map may be infinite or finite dimensional, i.e. $t$ can be infinity; it is usually a very large number (recall Definition \ref{definition:hilbert_space} where we said Hilbert space may have infinite number of dimensions).

% Considering both Eqs. \eqref{equation:mercer_kernel_decomposition} and \eqref{equation:feature_map_elements} shows that:
% \begin{align}
% k(x, y) = \langle \phi(x), \phi(y) \rangle_k = \phi(x)^\top \phi(y). \label{equation:kernel_inner_product}
% \end{align}
% Hence, the kernel between two points is the inner product of pulled data points to the feature space. Suppose we stack the feature maps of all points $\mathbf{X} \in \mathbb{R}^{d \times n}$ column-wise in:
% \begin{align}
% \mathbf{\Phi}(\mathbf{X}) := [\phi(x_1), \phi(x_2), \dots, \phi(x_n)], \label{equation:stacked_feature_maps}
% \end{align}
% which is $t \times n$ dimensional and $t$ may be infinity or a large number. The kernel matrix defined in Definition \ref{definition:gram_matrix} can be calculated as:
% \begin{align}
% \mathbb{R}^{n \times n} \ni \mathbf{K} = \langle \mathbf{\Phi}(\mathbf{X}), \mathbf{\Phi}(\mathbf{X}) \rangle_k = \mathbf{\Phi}(\mathbf{X})^\top \mathbf{\Phi}(\mathbf{X}). \label{equation:kernel_matrix_phi}
% \end{align}
% Eqs. \eqref{equation:kernel_inner_product} and \eqref{equation:kernel_matrix_phi} show that there is no need to compute kernel using eigenfunctions but a simple inner product suffices for kernel computation. This is the beauty of kernel methods which are simple to compute.

\end{document}