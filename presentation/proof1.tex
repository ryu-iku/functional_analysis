\documentclass[11pt, a4paper]{article}

% Use standard LaTeX packages compatible with pdflatex/latex engine
\usepackage[utf8]{inputenc}
\usepackage[T1]{fontenc}
\usepackage[english]{babel}
\usepackage{amsmath, amssymb, amsthm}
\usepackage{geometry}

% Remove redundant/conflicting geometry and documentclass
\geometry{a4paper, top=2.5cm, bottom=2.5cm, left=2cm, right=2cm}

% The modern font setup (fontspec, babelfont) has been removed to fix the compilation error.
% \usepackage{fontspec}
% \babelprovide[import, onchar=ids fonts]{english}
% \babelfont{rm}{Noto Sans}

% Define theorem and definition environments
\newtheorem{definition}{Definition}[section]
\newtheorem{theorem}{Theorem}[section]
\newtheorem{corollary}{Corollary}[section]
\newtheorem{lemma}{Lemma}[section]
\newtheorem{remark}{Remark}[section]

% Remove extra space from theorem definitions
\makeatletter
\def\thm@space@setup{%
  \thm@preskip=\topsep \thm@postskip=\topsep
  \itemsep \z@ \topsep \z@ \partopsep \z@ \parsep \z@
}
\makeatother

\begin{document}

\section{Mercer’s Theorem and Feature Map} \label{section:mercers_theorem_and_feature_map}

\subsection{Mercer’s Theorem} \label{section:mercers_theorem}

\begin{definition}[Definite Kernel] \label{definition:definite_kernel}
The function $k : [a, b] \times [a, b] \rightarrow \mathbb{R}$ is a definite kernel where the following double integral:
\begin{align}
J(f) = \int_{a}^{b} \int_{a}^{b} k(x, y) f(x) f(y) dx dy, \label{equation:definite_kernel_integral}
\end{align}
satisfies $J(f) > 0$ for all $f(x) \neq 0$.
\end{definition}

Mercer improved over Hilbert’s work to propose his theorem, the Mercer’s theorem, introduced in the following.

\begin{theorem}[Mercer’s Theorem] \label{theorem:mercers_theorem_main}
Suppose $k : [a, b] \times [a, b] \rightarrow \mathbb{R}$ is a continuous symmetric positive semi-definite kernel which is bounded:
\begin{align}
\sup_{x, y} k(x, y) < \infty. \label{equation:mercer_bounded}
\end{align}
Assume the operator $T_k$ takes a function $f(x)$ as its argument and outputs a new function as:
\begin{align}
T_k f(x) := \int_{a}^{b} k(x, y) f(y) dy, \label{equation:mercer_operator}
\end{align}
which is a Fredholm integral equation. The operator $T_k$ is called the Hilbert–Schmidt integral operator. This output function is positive semi-definite:
\begin{align}
\iint k(x, y) f(y) dx dy \geq 0. \label{equation:mercer_positive_semidefinite_integral}
\end{align}
Then, there is a set of orthonormal bases $\{ \psi_i(.) \}_{i=1}^\infty$ of $L^2(a, b)$ consisting of eigenfunctions of $T_K$ such that the corresponding sequence of eigenvalues $\{ \lambda_i \}_{i=1}^\infty$ are non-negative:
\begin{align}
\int k(x, y) \psi_i(y) dy = \lambda_i \psi_i(x). \label{equation:mercer_eigenfunction_decomposition}
\end{align}
The eigenfunctions corresponding to the non-zero eigenvalues are continuous on $[a, b]$ and $k$ can be represented as:
\begin{align}
k(x, y) = \sum_{i=1}^\infty \lambda_i \psi_i(x) \psi_i(y), \label{equation:mercer_kernel_decomposition}
\end{align}
where the convergence is absolute and uniform.
\end{theorem}

\begin{proof}
A roughly high-level proof for the Mercer’s theorem is as follows.

\textbf{Step 1 of proof:} According to assumptions of theorem, the Hilbert-Schmidt integral operator $T_k$ is a symmetric operator on $L^2(a, b)$ space. Consider a unit ball in $L^2(a, b)$ as input to the operator. As the kernel is bounded, $\sup_{x, y} k(x, y) < \infty$, the sequence $f_1, f_2, \dots$ converges in norm, i.e. $ \|f_n - f\| \rightarrow 0 $ as $n \rightarrow 0$. Therefore, according to the Arzel\`a-Ascoli theorem, the image of the unit ball after applying the operator is compact. In other words, the operator $T_k$ is compact.

\textbf{Step 2 of proof:} According to the spectral theorem, there exist several orthonormal bases $\{ \psi_i(.) \}_{i=1}^\infty$ in $L^2(a, b)$ for the compact operator $T_k$. This provides a spectral (or eigenvalue) decomposition for the operator $T_k$:
\begin{align}
T_k \psi_i(x) = \lambda_i \psi_i(x), \label{equation:spectral_decomposition_operator}
\end{align}
where $\{ \psi_i(.) \}_{i=1}^\infty$ and $\{ \lambda_i \}_{i=1}^\infty$ are the eigenvectors and eigenvalues of the operator $T_k$, respectively. Noticing the defined Eq. \eqref{equation:mercer_operator} and the eigenvalue decomposition, Eq. \eqref{equation:spectral_decomposition_operator}, we have:
\begin{align}
\int k(x, y) \psi_i(y) dy \overset{\eqref{equation:mercer_operator}}{=} T_k \psi_i(x) \overset{\eqref{equation:spectral_decomposition_operator}}{=} \lambda_i \psi_i(x). \label{equation:eigenfunction_decomposition_operator}
\end{align}
This proves the Eq. \eqref{equation:mercer_eigenfunction_decomposition} which is the eigenfunction decomposition of the operator $T_k$. Note that the eigenvectors $\{ \psi_i(.) \}_{i=1}^\infty$ are referred to as the \textit{eigenfunctions} because the decomposition is applied on a function or operator rather than a matrix. Note that eigenfunctions will be explained more in Section \ref{section:eigenfunctions}.

\textbf{Step 3 of proof:} According to Parseval’s theorem, the Bessel’s inequality can be converted to equality. For the orthonormal bases $\{ \psi_i(.) \}_{i=1}^\infty$ in the Hilbert space $\mathcal{H}$ associated with kernel $k$, we have for any function $f \in L^2(a, b)$:
\begin{align}
f = \sum_{i=1}^\infty \langle f, \psi_i \rangle_k \psi_i. \label{equation:parsevals_theorem}
\end{align}
If we replace $\psi_i$ with $f$ in Eq. \eqref{equation:spectral_decomposition_operator} and consider Eq. \eqref{equation:parsevals_theorem}, we will have:
\begin{align}
T_k f = \sum_{i=1}^\infty \lambda_i \langle f, \psi_i \rangle_k \psi_i. \label{equation:tk_f_lambda_psi}
\end{align}
One can consider Eq. \eqref{equation:mercer_operator} as $T_k f = kf$. Noticing this and Eq. \eqref{equation:tk_f_lambda_psi} results in:
\begin{align}
kf = \sum_{i=1}^\infty \lambda_i \langle f, \psi_i \rangle_k \psi_i. \label{equation:kf_lambda_psi}
\end{align}
Ignoring $f$ from Eq. \eqref{equation:kf_lambda_psi} gives:
\begin{align}
k(x, y) = \sum_{i=1}^\infty \lambda_i \psi_i(x) \psi_i(y), \label{equation:mercer_kernel_decomposition_reproved}
\end{align}
which is Eq. \eqref{equation:mercer_kernel_decomposition}; hence, that is proved.

\textbf{Step 4 of proof:} We define the truncated kernel $r_n$ (with parameter $n$) as:
\begin{align}
r_n(x, y) &:= k(x, y) - \sum_{i=1}^n \lambda_i \psi_i(x) \psi_i(y) \nonumber \\
&= \sum_{i=n+1}^\infty \lambda_i \psi_i(x) \psi_i(y). \label{equation:truncated_kernel}
\end{align}
As $T_k$ is an integral operator, this truncated kernel has positive kernel, i.e., for every $x \in [a, b]$, we have:
\begin{equation*} % Use equation* for the unnumbered part of the derivation
r_n(x, x) = k(x, x) - \sum_{i=1}^n \lambda_i \psi_i(x) \psi_i(x) \geq 0 
\end{equation*}
which implies
\begin{equation} % Use equation for the single line that requires the label
\sum_{i=1}^n \lambda_i \psi_i(x) \psi_i(x) \leq k(x, x) \leq \sup_{x \in [a, b]} k(x, x). \label{equation:truncated_kernel_bound}
\end{equation}
By Cauchy-Schwartz inequality, we have:
\begin{align*}
\left| \sum_{i=1}^n \lambda_i \psi_i(x) \psi_i(y) \right|^2 &\leq \left( \sum_{i=1}^n \lambda_i \psi_i(x) \psi_i(x) \right) \left( \sum_{i=1}^n \lambda_i \psi_i(y) \psi_i(y) \right) \\
&\overset{\eqref{equation:truncated_kernel_bound}}{\leq} \left( \sup_{x \in [a, b]} k(x, x) \right)^2.
\end{align*}
Taking second root from the sides of inequality gives:
\begin{align}
\sum_{i=1}^n \lambda_i \psi_i(x) \psi_i(x) \leq \sup_{x \in [a, b]} |k(x, x)| \overset{\eqref{equation:mercer_bounded}}{\leq} \infty. \label{equation:mercer_convergence}
\end{align}
This shows that the sequence $\sum_{i=1}^n \lambda_i \psi_i(x) \psi_i(x)$ converges absolutely and uniformly.
\end{proof}

\end{document}