\documentclass{article}
\usepackage{amsmath, amssymb}
\usepackage{geometry}
\geometry{a4paper, margin=1in}

\title{The Arzelà-Ascoli Theorem and its Application to Mercer's Theorem}
\author{}
\date{}

\begin{document}
\maketitle

The \textbf{Arzelà-Ascoli Theorem} is a fundamental result in real analysis and functional analysis that provides conditions under which a sequence of continuous functions on a compact set will have a uniformly convergent subsequence. It is a powerful tool for proving the existence of solutions to differential equations, properties of integral operators, and other results involving function spaces.

\section*{High-Level Intuition}

In finite-dimensional Euclidean space, a set is ``compact'' (meaning every sequence in the set has a convergent subsequence) if and only if it is \textbf{closed and bounded} (Heine-Borel theorem). However, in infinite-dimensional spaces, such as spaces of functions, being closed and bounded is not enough to guarantee compactness. Functions can be bounded but still ``wiggle'' infinitely fast, preventing convergence.

The Arzelà-Ascoli Theorem provides the \textbf{additional conditions} needed for a set of functions to be \textbf{relatively compact} (meaning its closure is compact) in a space of continuous functions. These conditions essentially limit how ``wiggly'' or ``spread out'' the functions in the set can be.

\section*{Formal Statement (Simplified for Context)}

Let $X$ be a compact metric space (e.g., a closed and bounded interval like $[a,b]$ in $\mathbb{R}$). Let $\mathcal{F}$ be a family (set) of continuous functions from $X$ to $\mathbb{R}$ (or $\mathbb{C}$, or $\mathbb{R}^n$).

The Arzelà-Ascoli Theorem states that $\mathcal{F}$ is \textbf{relatively compact} in the space of continuous functions $C(X)$ (equipped with the uniform norm, $\|f\|_\infty = \sup_{x \in X} |f(x)|$) if and only if it satisfies two conditions:

\begin{enumerate}
    \item \textbf{Pointwise Boundedness:} For each point $x \in X$, the set of values $\{f(x) : f \in \mathcal{F}\}$ is bounded. That is, for every $x \in X$, there exists a constant $M_x$ such that $|f(x)| \le M_x$ for all $f \in \mathcal{F}$.
    \begin{itemize}
        \item[\textit{Note:}] \textit{Often, in practical applications, a stronger condition called \textbf{uniform boundedness} is used: there exists a single constant $M$ such that $|f(x)| \le M$ for all $f \in \mathcal{F}$ and all $x \in X$. For continuous functions on a compact domain, pointwise boundedness implies uniform boundedness.}
    \end{itemize}

    \item \textbf{Equicontinuity:} The family $\mathcal{F}$ is equicontinuous. This means that all functions in the family are ``equally continuous'' at every point. Formally, for every $\epsilon > 0$ and every $x \in X$, there exists a $\delta > 0$ such that for all $f \in \mathcal{F}$ and all $x' \in X$ with $d(x, x') < \delta$, we have $|f(x) - f(x')| < \epsilon$.
    \begin{itemize}
        \item[\textit{Note:}] \textit{The key here is that $\delta$ depends only on $\epsilon$ and $x$, not on the individual function $f$. It ensures that no function in the family can vary too rapidly.}
    \end{itemize}
\end{enumerate}

If these two conditions are met, then every sequence of functions in $\mathcal{F}$ has a subsequence that converges uniformly to a continuous function.

\section*{Relevance to Mercer's Theorem (as per text)}

In \textbf{Step 1 of the proof of Mercer's Theorem}, the Arzelà-Ascoli Theorem is invoked to establish that the Hilbert-Schmidt integral operator $T_k$ is a \textbf{compact operator}.

The statement in the text is: ``Therefore, according to the Arzelà-Ascoli theorem (Arzelà, 1895), the image of the unit ball after applying the operator is compact. In other words, the operator $T_k$ is compact.''

Let's break down how this connection works:

\begin{enumerate}
    \item \textbf{The Operator $T_k$:}
    $$T_k f(x) := \int_a^b k(x,y) f(y) \, dy \quad \text{(Eq. 18)}$$
    This operator maps functions $f \in L^2([a,b])$ to functions $T_k f \in C([a,b])$ (the space of continuous functions on $[a,b]$), provided $k(x,y)$ is continuous.

    \item \textbf{Compact Operator Definition:} A linear operator $A: V \to W$ (between normed spaces) is called \textbf{compact} if it maps bounded sets in $V$ to relatively compact sets in $W$. In the context of Mercer's Theorem, $V = L^2([a,b])$ and $W = C([a,b])$. We want to show that the image of the unit ball in $L^2([a,b])$ (i.e., the set $\mathcal{F} = \{T_k f \mid \|f\|_{L^2} \le 1\}$) is relatively compact in $C([a,b])$.

    \item \textbf{Applying Arzelà-Ascoli:} To show that the set $\mathcal{F} = \{T_k f \mid \|f\|_{L^2} \le 1\}$ is relatively compact in $C([a,b])$, we need to verify the two conditions of Arzelà-Ascoli:

    \begin{itemize}
        \item \textbf{Uniform Boundedness:} We need to show that there is a constant $M$ such that for all $f$ with $\|f\|_{L^2} \le 1$, and for all $x \in [a,b]$, $|T_k f(x)| \le M$.
        \begin{align*}
        |T_k f(x)| &= \left| \int_a^b k(x,y) f(y) \, dy \right| \\
        &\le \left( \int_a^b |k(x,y)|^2 \, dy \right)^{1/2} \left( \int_a^b |f(y)|^2 \, dy \right)^{1/2} \quad \text{(Cauchy-Schwarz)} \\
        &\le \left( \int_a^b |k(x,y)|^2 \, dy \right)^{1/2} \cdot 1 \quad \text{(Since } \|f\|_{L^2} \le 1 \text{)}
        \end{align*}
        The kernel $k(x,y)$ is 
        % assumed to be \textbf{continuous} on the compact set $[a,b] \times [a,b]$, so it is 
        bounded:
        % (Eq. 17: 
        $\sup_{x,y} k(x,y) < \infty$.
        % ). 
        Let $K_{max} = \sup_{x,y} |k(x,y)|$.
        Then 
        \begin{align*}
            \int_a^b |k(x,y)|^2 \, dy \le \int_a^b K_{max}^2 \, dy = K_{max}^2 (b-a)
            \\
            |T_k f(x)| \le K_{max} \sqrt{b-a}.
        \end{align*}
        % So, . 
        % This shows that the family $\{T_k f\}$ is uniformly bounded.

        \item \textbf{Equicontinuity:} We need to show that for any $\epsilon > 0$, there exists a $\delta > 0$ such that for all $f$ with $\|f\|_{L^2} \le 1$, and for any $x_1, x_2 \in [a,b]$ with $|x_1 - x_2| < \delta$, we have $|T_k f(x_1) - T_k f(x_2)| < \epsilon$.
        \begin{align*}
        |T_k f(x_1) - T_k f(x_2)| &= \left| \int_a^b (k(x_1,y) - k(x_2,y)) f(y) \, dy \right| \\
        &\le \left( \int_a^b |k(x_1,y) - k(x_2,y)|^2 \, dy \right)^{1/2} \left( \int_a^b |f(y)|^2 \, dy \right)^{1/2} \quad \text{(Cauchy-Schwarz)} \\
        &\le \left( \int_a^b |k(x_1,y) - k(x_2,y)|^2 \, dy \right)^{1/2} \quad \text{(Since } \|f\|_{L^2} \le 1 \text{)}
        \end{align*}

        Since $k(x,y)$ is \textbf{continuous} on the compact set $[a,b] \times [a,b]$, it is \textbf{uniformly continuous}.
        \begin{align*}
            \forall \eta > 0,
            \exists \delta > 0, \text{ s.t. }  
            |x_1 - x_2| < \delta 
            \implies 
            \forall y \in [a,b], 
            |k(x_1,y) - k(x_2,y)| < \eta,
        \end{align*}

        So, if $|x_1 - x_2| < \delta$, then 
        \begin{align*}
            \int_a^b |k(x_1,y) - k(x_2,y)|^2 \, dy &< \int_a^b \eta^2 \, dy = \eta^2 (b-a).
            \\
            |T_k f(x_1) - T_k f(x_2)| &< \eta \sqrt{b-a}
        \end{align*}

        % Thus, . By choosing $\eta$ appropriately (e.g., $\eta = \epsilon/\sqrt{b-a}$), we can make this difference less than $\epsilon$. Crucially, $\delta$ depends only on $\epsilon$ (and the uniform continuity of $k$), not on $f$. This demonstrates \textbf{equicontinuity}.
    \end{itemize}

\end{enumerate}

Since both uniform boundedness and equicontinuity are satisfied, the Arzelà-Ascoli Theorem ensures that the image of the unit ball under $T_k$ is relatively compact in $C([a,b])$. This implies that $T_k$ is a \textbf{compact operator}. The compactness of $T_k$ is a critical prerequisite for applying the Spectral Theorem for compact self-adjoint operators in the next step of Mercer's proof.



\end{document}