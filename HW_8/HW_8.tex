\documentclass{article}
\usepackage{amsmath, amssymb, amsthm}
\usepackage{geometry}
\geometry{a4paper, margin=1in}

\DeclareMathOperator{\esssup}{ess\,sup}
\newcommand{\norm}[1]{\left\|#1\right\|}
\newcommand{\abs}[1]{\left|#1\right|}
\newcommand{\set}[1]{\left\{#1\right\}}
\newcommand{\R}{\mathbb{R}}
\newcommand{\C}{\mathbb{C}}
\newcommand{\N}{\mathbb{N}}
\newcommand{\ellp}{\ell^p}
\newcommand{\ellq}{\ell^q}
\newcommand{\ip}[2]{\left\langle #1, #2 \right\rangle}

\newtheorem{problem}{Problem}

\begin{document}

\begin{center}
    \textbf{\Large Functional Analysis - Homework 8}
\end{center}

\hrule
\vspace{0.5em}

% ---------------------------
% Problem 1
% ---------------------------
\begin{problem}
\end{problem}
% ------------- Solution

To show the inequality, for \( 2 \le p < \infty \),
\begin{align}
\left|\frac{a+b}{2}\right|^{p} + \left|\frac{a-b}{2}\right|^{p}
    \le \frac{1}{2}\left(|a|^{p} + |b|^{p}\right).
\end{align}
, it is sufficient to prove the real-variable inequality for \( t \in [0,1] \):
\begin{align}
\left(\frac{1+t}{2}\right)^{p} + \left(\frac{1-t}{2}\right)^{p}
    \le \frac{1+t^{p}}{2}.
\end{align}

Define the function \( f(t) \) on \( [0,1] \) by
\begin{align}
f(t) = \frac{1}{2}\left(1 + t^{p}\right)
       - \left[
           \left(\frac{1+t}{2}\right)^{p}
           + \left(\frac{1-t}{2}\right)^{p}
         \right].
\end{align}
We aim to show that \( f(t) \ge 0 \).

At \( t = 1 \):
\begin{align}
f(1)
    &= \frac{1}{2}(1 + 1) - \left[ 1^{p} + 0^{p} \right] \notag \\
    &= 1 - 1 = 0.
\end{align}

At \( t = 0 \):
\begin{align}
f(0)
    &= \frac{1}{2}(1) - \left[ 2^{-p} + 2^{-p} \right] \notag \\
    &= \frac{1}{2} - 2^{-(p-1)}.
\end{align}
Since \( p \ge 2 \) implies \( 2^{-(p-1)} \le \tfrac{1}{2} \),
it follows that \( f(0) \ge 0 \).

\begin{align}
f'(t)
    = \frac{p}{2}t^{p-1}
        - p\,2^{-p}\!\left[(1+t)^{p-1} - (1-t)^{p-1}\right].
\end{align}
We verify that \( f'(1) = 0 \):
\begin{align}
f'(1)
    &= \frac{p}{2} - p\,2^{-p}\left[2^{p-1} - 0\right] \notag \\
    &= \frac{p}{2} - \frac{p}{2} = 0.
\end{align}

\begin{align}
f''(t)
    = \frac{p(p-1)}{2}
        \left[
        t^{p-2}
        - 2^{-(p-1)}\!\left((1+t)^{p-2} + (1-t)^{p-2}\right)
        \right].
\end{align}

The function \(\psi(s)=s^{p-2}\) is convex on \([0,\infty)\) when \(p\ge 2\)
(since \(p-2\ge 0\)). Set
\[
x_1=\frac{1+t}{2},\qquad x_2=\frac{1-t}{2},
\]
so that \(x_1,x_2\in[0,1]\) and \(x_1+x_2=1\). Writing \(t = x_1-x_2\) and using the convexity
of \(\psi\) together with the evenness of the power \((\cdot)^{p-2}\), one obtains the pointwise
inequality (for \(t\in[0,1]\))
\begin{align}
    t^{\,p-2} \le 2^{-(p-1)}\bigl((1+t)^{p-2} + (1-t)^{p-2}\bigr),
\end{align}
which implies \(f''(t)\le 0\) for \(t\in[0,1]\) when \(p>2\) (and the boundary case \(p=2\)
may be checked directly). 

Thus \(f''(t)\le 0\) on \([0,1]\), so \(f\) is concave on \([0,1]\). Since a concave function
attains its minimum at the boundary, and we already have \(f(0)\ge 0\) and \(f(1)=0\),
it follows that \(f(t)\ge 0\) for all \(t\in[0,1]\).

Hence, the scalar inequality is proven.
By integrating this result over the measure space
\( (\Omega, \mathcal{F}, \mu) \), we obtain
\begin{align}
\int_{\Omega}
\left[
\left|\frac{f(x)+g(x)}{2}\right|^{p}
+
\left|\frac{f(x)-g(x)}{2}\right|^{p}
\right] d\mu
&\le
\int_{\Omega}
\frac{1}{2}\left(|f(x)|^{p} + |g(x)|^{p}\right) d\mu, \\
\left\|\frac{f+g}{2}\right\|_{L_{p}(\mu)}^{p}
+
\left\|\frac{f-g}{2}\right\|_{L_{p}(\mu)}^{p}
&\le
\frac{1}{2}\left(\|f\|_{L_{p}(\mu)}^{p} + \|g\|_{L_{p}(\mu)}^{p}\right).
\end{align}

This inequality implies that
\( L_{p}(\mu) \) is uniformly convex for \( p \ge 2 \).


\hrule
\vspace{0.5em}

% ---------------------------
% Problem 2
% ---------------------------

\begin{problem}

\end{problem}
% ------------- Solution

\begin{enumerate}
\item[\textbf{(1)}] \textbf{T is an isometry:} \(\|Tf\|_{(L_q)^*} = \|f\|_{L_p}\).

By definition,
\begin{align}
\|Tf\|_{(L_q)^*}
= \sup_{\|g\|_{L_q} \le 1} |\langle Tf, g \rangle|
= \sup_{\|g\|_{L_q} \le 1} \left|\int fg \, d\mu\right|.
\end{align}
Hölder’s inequality gives
\begin{align}
\left|\int fg\,d\mu\right|
\le \|f\|_{L_p}\|g\|_{L_q},
\end{align}
so \(\|Tf\|_{(L_q)^*} \le \|f\|_{L_p}\).

For equality, assume \(f \neq 0\) and set
\begin{align}
g_0(x) = sgn(f(x))\left(\frac{|f(x)|}{\|f\|_{L_p}}\right)^{p-1}.
\end{align}
Then
\begin{align}
\|g_0\|_{L_q}^q
= \int |g_0|^q \, d\mu
= \frac{1}{\|f\|_{L_p}^{p}} \int |f|^{p} \, d\mu = 1,
\end{align}
so \(\|g_0\|_{L_q} = 1\). Moreover,
\begin{align}
\langle Tf, g_0 \rangle
= \int f g_0 \, d\mu
= \frac{1}{\|f\|_{L_p}^{p-1}} \int |f|^p \, d\mu
= \|f\|_{L_p}.
\end{align}
Hence \(\|Tf\|_{(L_q)^*} = \|f\|_{L_p}\); thus \(T\) is an isometry.

\item[\textbf{(2)}] \textbf{\(T(L_p)\) is closed in \((L_q)^*\).}

Let \(\{Tf_n\}\) be a Cauchy sequence in \(T(L_p)\).
Since \(T\) is an isometry,
\begin{align}
\|Tf_n - Tf_m\|_{(L_q)^*} = \|f_n - f_m\|_{L_p},
\end{align}
so \(\{f_n\}\) is Cauchy in \(L_p\).
Because \(L_p\) is complete, \(f_n \to f \in L_p\), and by continuity of \(T\),
\begin{align}
Tf_n \to Tf.
\end{align}
Thus the limit lies in \(T(L_p)\), proving that \(T(L_p)\) is closed in \((L_q)^*\).

\item[\textbf{(3)}] \textbf{\(L_p\) is reflexive for \(1 < p \le 2\).}

For \(1 < p \le 2\), we have \(q \ge 2\).
By Clarkson’s inequality, \(L_q(\mu)\) is uniformly convex for \(q \ge 2\).
 Since every uniformly convex Banach space is reflexive, it is reflexive.

A dual space is reflexive if and only if its predual is reflexive,
so \((L_q)^*\) is reflexive.

Closed subspaces of reflexive spaces are reflexive; thus \(T(L_p)\) is reflexive.
Since \(T\) is an isometric isomorphism between \(L_p\) and \(T(L_p)\),
it follows that \(L_p\) itself is reflexive.


\end{enumerate}


\hrule
\vspace{0.5em}

% ---------------------------
% Problem 3
% ---------------------------
\begin{problem}
\end{problem}
% ------------- Solution

a)
Let $H$ be a Hilbert space with inner product $\langle \cdot,\cdot\rangle$ and norm 
$\norm{x} = \sqrt{\langle x,x\rangle}$.
We need to show: for every $\varepsilon \in (0,2]$, there exists $\delta > 0$ such that
for all $x,y \in H$ with $\norm{x} \le 1$, $\norm{y} \le 1$, and $\norm{x-y} \ge \varepsilon$, such that
\begin{align}
    \norm{\frac{x+y}{2}} \le 1 - \delta.
\end{align}

In Hilbert spaces,
\begin{align}
    \norm{x+y}^2 + \norm{x-y}^2 = 2(\norm{x}^2 + \norm{y}^2).
    \\
    \norm{\frac{x+y}{2}}^2
    = \frac{\norm{x}^2 + \norm{y}^2}{2} - \frac{\norm{x-y}^2}{4}.
\end{align}

Since $\norm{x}, \norm{y} \le 1$, we have
$\frac{\norm{x}^2 + \norm{y}^2}{2} \le 1$,
and $\norm{x-y} \ge \varepsilon$ gives
$-\frac{\norm{x-y}^2}{4} \le -\frac{\varepsilon^2}{4}$.

Thus
\begin{align}
    \norm{\frac{x+y}{2}}^2
    \le 1 - \frac{\varepsilon^2}{4}.
    \\
    \norm{\frac{x+y}{2}} \le \sqrt{1 - \frac{\varepsilon^2}{4}}.
\end{align}
Set
\begin{align}
    \delta = 1 - \sqrt{1 - \frac{\varepsilon^2}{4}} > 0.
\end{align}
Then
\begin{align}
\norm{\frac{x+y}{2}} \le 1 - \delta,
\end{align}
which implies that every Hilbert space is uniformly convex.



b) \(L_\infty\) is not uniformly convex.

Let $X = L_\infty[0,2]$ with the norm
$\norm{f}_\infty$ as the  essential supremum,
and consider the unit ball $B_1 = \{ f \in X : \norm{f}_\infty \le 1 \}$.

Define
\begin{align}
x(t) = 1, \qquad
y(t) = 
\begin{cases}
1, & 0 \le t \le 1,\\
-1, & 1 < t \le 2.
\end{cases}
\end{align}
Then $\norm{x}_\infty = \norm{y}_\infty = 1$, so $x,y \in B_1$.
Compute:
\begin{align}
(x - y)(t) =
\begin{cases}
0, & 0 \le t \le 1,\\
2, & 1 < t \le 2,
\end{cases}
\qquad
\norm{x - y}_\infty = 2.
\end{align}
Choose $\varepsilon = 1.9 < 2$, so $\norm{x - y}_\infty > \varepsilon$.

The midpoint satisfies
\begin{align}
\frac{x(t) + y(t)}{2} =
\begin{cases}
1, & 0 \le t \le 1,\\
0, & 1 < t \le 2,
\end{cases}
\qquad
\norm{\frac{x+y}{2}}_\infty = 1.
\end{align}
Hence, for this pair $x,y$,
$\norm{x-y}_\infty > \varepsilon$
but
$\norm{\frac{x+y}{2}}_\infty = 1$,
so for any $\delta > 0$,
$\norm{\frac{x+y}{2}}_\infty \not< 1 - \delta$.

Therefore, no $\delta>0$ satisfies the definition of uniform convexity,
and $L_\infty$ is not uniformly convex.



\hrule
\vspace{0.5em}

\end{document}