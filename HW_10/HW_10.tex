\documentclass{article}
\usepackage{amsmath, amssymb, amsthm}
\usepackage{geometry}
\geometry{a4paper, margin=1in}

\DeclareMathOperator{\esssup}{ess\,sup}
\newcommand{\norm}[1]{\left\|#1\right\|}
\newcommand{\abs}[1]{\left|#1\right|}
\newcommand{\set}[1]{\left\{#1\right\}}
\newcommand{\R}{\mathbb{R}}
\newcommand{\C}{\mathbb{C}}
\newcommand{\N}{\mathbb{N}}
\newcommand{\ellp}{\ell^p}
\newcommand{\ellq}{\ell^q}
\newcommand{\ip}[2]{\left\langle #1, #2 \right\rangle}
\newcommand{\re}{\operatorname{Re}}
\newcommand{\dist}{\operatorname{dist}}
\newcommand{\Res}{\operatorname{Res}}

% Custom commands for cleaner notation
\newcommand{\Ltwo}{L^2(\Omega)}
\newcommand{\LtwoX}{L^2(\Omega \times \Omega)}
\newcommand{\opnorm}[1]{\lVert#1\rVert_{\mathcal{L}}} % Operator norm
\newcommand{\Ltwodomain}{L^{2}([a,b];\C)}
\newcommand{\lnorm}[1]{\norm{#1}_{L^2([a,b])}} % L^2 norm using the new \norm command
\newcommand{\Linfdomain}{L^{\infty}([a,b])}

% Define a proof environment
% \newtheorem*{proofpart}{Proof}
% \renewenvironment{proof}{\begin{proofpart}}{\qed\end{proofpart}}
% \DeclareMathOperator{\Span}{span}

\newtheorem{problem}{Problem}

\begin{document}

\begin{center}
    \textbf{\Large Functional Analysis - Homework 10}
\end{center}

\hrule
\vspace{0.5em}

% ---------------------------
% Problem 1
% ---------------------------
\begin{problem}
\end{problem}


\subsection*{a)}
\begin{proof}
\textbf{($\Rightarrow$)} 
Assume $A$ is compact. 

Let $(x_n)$ be a bounded sequence, so $\norm{x_n} \leq M$ for some $M > 0$.
Define $z_n = x_n/M$. Then $\norm{z_n} \leq 1$, so $z_n \in B_1(0)$.

$(Az_n) \subset A(B_1(0))$ is relatively compact since $A$ is compact.
Thus, $(Az_n)$ has a convergent subsequence $(Az_{n_k})$.
Since $Ax_{n_k} = A(M z_{n_k}) = M \cdot Az_{n_k}$, the sequence $(Ax_{n_k})$ also converges in $Y$.

---

\textbf{($\Leftarrow$)} Assume the sequential property holds. Let $(y_n)$ be an arbitrary sequence in $K = \overline{A(B_1(0))}$.
By the definition of closure, for each $y_n$, there exists $z_n \in A(B_1(0))$ such that $\norm{y_n - z_n}_Y < 1/n$.
Since $z_n \in A(B_1(0))$, there is $x_n \in B_1(0)$ (hence $\norm{x_n} \leq 1$) such that $z_n = Ax_n$.
The sequence $(x_n)$ is bounded. By hypothesis, $(x_n)$ has a subsequence $(x_{n_k})$ such that $(Ax_{n_k})$ converges to some $y_0 \in Y$. Let $z_{n_k} = Ax_{n_k}$.
Consider the corresponding subsequence $(y_{n_k})$.
\begin{align}
    \norm{y_{n_k} - y_0}_Y \leq \norm{y_{n_k} - z_{n_k}}_Y + \norm{z_{n_k} - y_0}_Y
\end{align}
As $k \to \infty$, $\norm{y_{n_k} - z_{n_k}}_Y < 1/n_k \to 0$ and $\norm{z_{n_k} - y_0}_Y \to 0$.
Thus, $(y_{n_k})$ converges to $y_0$. Since every sequence in $K$ has a convergent subsequence, $K$ is compact, and $A$ is a compact operator.
\end{proof}

\subsection*{b)}

\begin{proof}
\textbf{Subspace Property:}
\begin{enumerate}
    \item \textbf{Scalar Multiplication:} If $A \in \mathcal{K}(X,Y)$, then $\overline{A(B_1(0))}$ is compact. For $\alpha \in \mathbb{R}$, $(\alpha A)(B_1(0)) = \alpha \cdot A(B_1(0))$. Since multiplication by a scalar is continuous, the continuous image of a compact set is compact. Hence $\overline{(\alpha A)(B_1(0))} = \alpha \overline{A(B_1(0))}$ is compact, and $\alpha A \in \mathcal{K}(X,Y)$.
    \item \textbf{Addition:} If $A, B \in \mathcal{K}(X,Y)$, then $\overline{A(B_1(0))}$ and $\overline{B(B_1(0))}$ are compact. The set $(A+B)(B_1(0)) \subset A(B_1(0)) + B(B_1(0))$. The set $K_A+K_B$, where $K_A = \overline{A(B_1(0))}$ and $K_B = \overline{B(B_1(0))}$, is the continuous image of the compact set $K_A \times K_B$ under vector addition, hence $K_A+K_B$ is compact. Since $\overline{(A+B)(B_1(0))}$ is a closed subset of a compact set $K_A+K_B$, it is compact. Thus $A+B \in \mathcal{K}(X,Y)$.
\end{enumerate}

\textbf{Closed Property:} Assume $Y$ is complete. Let $(A_n)$ be a sequence in $\mathcal{K}(X, Y)$ such that $A_n \to A$ in the operator norm $\mathcal{L}(X, Y)$. We show $A$ is compact using total boundedness.

Let $\epsilon > 0$. Since $A_n \to A$, there exists $N$ such that $\norm{A_N - A} < \epsilon$.
Since $A_N$ is compact, the set $K_N = A_N(B_1(0))$ is relatively compact, and thus totally bounded. $K_N$ has a finite $\epsilon$-net $\{y_1, \dots, y_M\}$.

Now, consider $K = A(B_1(0))$. For any $y = Ax \in K$ (with $\norm{x} \leq 1$), consider $y_N = A_N x \in K_N$.
Since $\{y_1, \dots, y_M\}$ is an $\epsilon$-net for $K_N$, there is $y_i$ such that $\norm{y_N - y_i} < \epsilon$.
By the triangle inequality:
\begin{align}
\norm{y - y_i} 
&\leq \norm{Ax - A_N x} + \norm{A_N x - y_i}
\\
\norm{Ax - A_N x} &\leq \norm{A - A_N} \norm{x} < \epsilon \cdot 1 = \epsilon \\
\norm{y - y_i} &< \epsilon + \epsilon = 2 \epsilon
\end{align}
Thus, $\{y_1, \dots, y_M\}$ is an $\epsilon$-net for $K$. Since $K$ is totally bounded, its closure $\overline{K}$ is compact (as $Y$ is complete). Hence $A \in \mathcal{K}(X,Y)$, proving $\mathcal{K}(X,Y)$ is closed.
\end{proof}

\subsection*{c) }
\begin{proof}
Since $A(X)$ is a finite-dimensional subspace of $Y$, it is a closed set in $Y$.
The set $K = \overline{A(B_1(0))}$ is a subset of $A(X)$, and since $A(X)$ is closed, we have $K \subset A(X)$.
Also, $A$ is a bounded operator, so $K$ is a bounded set.
Since $K$ is a closed and bounded set contained in the finite-dimensional space $A(X)$, $K$ is compact.
Thus, $A$ is a compact operator.
\end{proof}

\subsection*{d)}

\begin{proof}
\textbf{Case 1: $A \in \mathcal{K}(X,Y)$, $B \in \mathcal{L}(Y,Z)$.}
Since $A$ is compact, $K_A = \overline{A(B_1(0))}$ is a compact set in $Y$.
Since $B$ is bounded, it is continuous. The image of the compact set $K_A$ under $B$, $B(K_A)$, is compact in $Z$.
Since $(B \circ A)(B_1(0)) \subset B(A(B_1(0))) \subset B(K_A)$, the image of the unit ball under $B \circ A$ is relatively compact. Thus $B \circ A$ is compact.

\textbf{Case 2: $A \in \mathcal{L}(X,Y)$, $B \in \mathcal{K}(Y,Z)$.}
Since $A$ is bounded, $A(B_1(0))$ is a bounded set in $Y$. Let $M = \norm{A}$. Then $A(B_1(0)) \subset M \cdot B_1(0)$ in $Y$.
The image of the unit ball under $B \circ A$ satisfies:
\begin{align}
(B \circ A)(B_1(0)) = B(A(B_1(0))) \subset B(M \cdot B_1(0)) = M \cdot B(B_1(0))
\end{align}
The closure $\overline{(B \circ A)(B_1(0))} \subset M \cdot \overline{B(B_1(0))}$.
Since $B$ is compact, $\overline{B(B_1(0))}$ is compact. Since scaling by $M$ is continuous, $M \cdot \overline{B(B_1(0))}$ is compact.
Thus, $\overline{(B \circ A)(B_1(0))}$ is a closed subset of a compact set, hence it is compact, and $B \circ A$ is compact.
\end{proof}

\subsection*{e) }

\begin{proof}
Let $(x_n)$ be an arbitrary bounded sequence in $X$.
Since $X$ is a reflexive space, the Eberlein-Smulian Theorem implies that the bounded set $\{x_n\}$ is weakly sequentially compact.
Therefore, $(x_n)$ has a weakly convergent subsequence $(x_{n_k})$, which converges to some $x_0 \in X$:
$$
x_{n_k} \rightharpoonup x_0 \quad \text{in } X
$$
By the hypothesis on the operator $A$, it maps this weakly convergent sequence to a norm-convergent sequence in $Y$:
\begin{align}
\norm{Ax_{n_k} - Ax_0}_Y \to 0 \quad \text{as } k \to \infty
\end{align}
Since the sequence of images $(Ax_{n_k})$ is norm-convergent, it satisfies the condition required by statement (a). Therefore, $A$ is a compact operator.
\end{proof}

\vspace{0.5em}
\hrule
\vspace{0.5em}

% ---------------------------
% Problem 2
% ---------------------------

\begin{problem}
\end{problem}
% ------------- Solution

\subsection*{a)}

\begin{proof}

For any $f \in \Ltwo$:
\begin{equation} \label{eq:Kf-norm-squared}
    \norm{Kf}_{\Ltwo}^2 = \int_{\Omega} \left| \int_{\Omega} k(x,y)f(y)dy \right|^2 dx
\end{equation}
Applying the Cauchy-Schwarz inequality to the inner integral (with respect to $y$):
\begin{align} \label{eq:cauchy-schwarz}
    \left| \int_{\Omega} k(x,y)f(y)dy \right|^2 &\leq \left( \int_{\Omega} |k(x,y)|^2 dy \right) \left( \int_{\Omega} |f(y)|^2 dy \right) \\
    &= \left( \int_{\Omega} |k(x,y)|^2 dy \right) \norm{f}_{\Ltwo}^2
\end{align}
Substituting this back into Equation \eqref{eq:Kf-norm-squared}:
\begin{align} \label{eq:l2-bound}
    \norm{Kf}_{\Ltwo}^2 &\leq \int_{\Omega} \left[ \left( \int_{\Omega} |k(x,y)|^2 dy \right) \norm{f}_{\Ltwo}^2 \right] dx \\
    &= \norm{f}_{\Ltwo}^2 \int_{\Omega} \int_{\Omega} |k(x,y)|^2 dy dx \\
    &= \norm{k}_{\LtwoX}^2 \norm{f}_{\Ltwo}^2
\end{align}
Taking the square root gives the operator norm bound:
\begin{equation} \label{eq:operator-bound}
    \norm{Kf}_{\Ltwo} \leq \norm{k}_{\LtwoX} \norm{f}_{\Ltwo}
\end{equation}
Since $k \in \LtwoX$, $\norm{k}_{\LtwoX} < \infty$. Thus, $\norm{Kf}_{\Ltwo} < \infty$, which proves that $Kf \in \Ltwo$

% , and that $K$ is a bounded operator in $\mathcal{L}(\Ltwo, \Ltwo)$. The operator norm is $\opnorm{K} \leq \norm{k}_{\LtwoX}$.
\end{proof}


\subsection*{b)}

\begin{proof}
% We prove $K$ is compact by showing it is the limit of a sequence of finite-rank operators in the operator norm.

% \begin{enumerate}
%     \item Approximation of the Kernel:
    Let $\{e_i\}_{i=1}^\infty$ be a complete orthonormal basis for $\Ltwo$. Then the tensor products $\{e_i(x)e_j(y)\}_{i,j=1}^\infty$ form an orthonormal basis for $\LtwoX$. Since $k \in \LtwoX$, we can approximate it by its partial Fourier sums. Define the approximating kernels $k_n$:
    \begin{equation} \label{eq:kernel-approx}
    k_n(x,y) = \sum_{i=1}^n \sum_{j=1}^n c_{ij} e_i(x) e_j(y)
    \end{equation}
    This approximation converges in the $L^2$ norm: $\norm{k - k_n}_{\LtwoX} \to 0$ as $n \to \infty$.

    % \item Define Finite-Rank Operators ($K_n$):
    Define the corresponding integral operators $K_n$:
    \begin{align} \label{eq:kn-operator}
    (K_n f)(x) &= \int_{\Omega} k_n(x,y)f(y)dy \\
    &= \int_{\Omega} \left( \sum_{i=1}^n \sum_{j=1}^n c_{ij} e_i(x) e_j(y) \right) f(y)dy \\
    &= \sum_{i=1}^n e_i(x) \left( \sum_{j=1}^n c_{ij} \int_{\Omega} e_j(y)f(y)dy \right)
    \end{align}
    The range of $K_n$ is spanned by the finite set $\{e_1, \dots, e_n\}$, hence $K_n$ is a finite-rank operator and thus a compact operator for all $n$.

    % \item Show Norm Convergence:
    We bound the operator norm of the difference $K - K_n$ using the result from part (a) (Equation \eqref{eq:operator-bound}):
    \begin{align} \label{eq:norm-convergence}
    \opnorm{K - K_n} &= \sup_{\norm{f}_{\Ltwo}\leq 1} \norm{(K - K_n)f}_{\Ltwo} \\
    &= \sup_{\norm{f}_{\Ltwo}\leq 1} \norm{\int_{\Omega} (k(x,y) - k_n(x,y)) f(y)dy}_{\Ltwo} \\
    &\leq \norm{k - k_n}_{\LtwoX} \cdot \sup_{\norm{f}_{\Ltwo}\leq 1} \norm{f}_{\Ltwo} \\
    &= \norm{k - k_n}_{\LtwoX}
    \end{align}
    Since $\norm{k - k_n}_{\LtwoX} \to 0$ as $n \to \infty$, we have $\opnorm{K - K_n} \to 0$.

    % \item Conclude Compactness:
    The sequence of compact operators $(K_n)$ converges to $K$ in the operator norm. Since $\Ltwo$ is a Hilbert space, the set of compact operators $\mathcal{K}(\Ltwo, \Ltwo)$ is a closed subspace of $\mathcal{L}(\Ltwo, \Ltwo)$. Therefore, the limit operator $K$ must be compact.
% \end{enumerate}
\end{proof}


\vspace{0.5em}
\hrule
\vspace{0.5em}

% ---------------------------
% Problem 3
% ---------------------------

\begin{problem}
\end{problem}
% ------------- Solution

\subsection*{a)}

% The operator $A$ is continuous (bounded) if there exists $C \ge 0$ such that $\lnorm{Af} \le C \lnorm{f}$ for all $f \in \Ltwodomain$.

\begin{enumerate}
    \item \textbf{Continuity Proof:}
    % We compute the square of the $L^2$ norm of $Af$:
    \begin{equation} \label{eq:A-norm-squared}
    \lnorm{Af}^2 = \int_a^b \abs{x^2 f(x)}^2 dx = \int_a^b x^4 \abs{f(x)}^2 dx
    \end{equation}
    Since $a \le 0 \le b$, the maximum value of $x^2$ on the interval $[a,b]$ is $M^2 = \max(a^2, b^2)$.
    For all $x \in [a,b]$, we have $x^2 \le M^2$, which implies $x^4 \le M^4$.

    Substituting this bound into Equation \eqref{eq:A-norm-squared}:
    \begin{align} \label{eq:continuity-bound}
    \lnorm{Af}^2 &\le \int_a^b M^4 \abs{f(x)}^2 dx \\
    &= M^4 \int_a^b \abs{f(x)}^2 dx = M^4 \lnorm{f}^2
    \end{align}
    Taking the square root, $\lnorm{Af} \le M^2 \lnorm{f}$. Since $M^2 = \max(a^2, b^2)$ is finite, $A$ is continuous.

    \item \textbf{Operator Norm $||A||$:}
    The operator norm is defined as $||A|| = \sup_{\lnorm{f}=1} \lnorm{Af}$.
    For a multiplication operator $A$ by $g(x)=x^2$ on $L^2$, its norm is the essential supremum of the multiplier function $g$:
    \begin{equation}
    \norm{A} = \norm{x^2}_{\Linfdomain} = \esssup_{x \in [a,b]} \abs{x^2} = \max_{x \in [a,b]} x^2 = \max(a^2, b^2).
    \end{equation}
\end{enumerate}



\subsection*{b)}

\begin{proof}
A complex number $\lambda$ is an eigenvalue if there exists a non-zero function $f \in \Ltwodomain$ such that $Af = \lambda f$.
The eigenvalue equation is:
\begin{equation} \label{eq:eigenvalue-equation}
x^2 f(x) = \lambda f(x) \quad \text{a.e. } x \in [a,b]
\end{equation}
Thus,
$$(x^2 - \lambda) f(x) = 0 \quad \text{a.e. } x \in [a,b]$$
, which implies that $f(x)$ can be non-zero only on the set $E_\lambda = \set{x \in [a,b] \mid x^2 = \lambda}$.

The set $E_\lambda$ contains at most two distinct roots, which means its Lebesgue measure is zero:
$$m(E_\lambda) = 0$$
If $f \in \Ltwodomain$ is an eigenfunction, it must be zero on the complement of $E_\lambda$. Since $m(E_\lambda)=0$, $f$ is zero almost everywhere on the entire interval $[a,b]$.
In $L^2$ space, this means $f$ is the zero function ($f=0$).
Since there is no non-zero eigenfunction, the operator $A$ has no eigenvalues.
\end{proof}



\subsection*{c)}

For a multiplication operator $A_g f = g f$ on $L^2$, the spectrum $\sigma(A_g)$ is the essential range of the multiplier function $g$. Here the multiplier is $g(x) = x^2$.

\begin{enumerate}
    \item \textbf{Essential Range of $x^2$:}
    Since $x^2$ is continuous on the compact interval $[a,b]$, its essential range is the closed interval:
    \begin{equation}
    R = [\min_{x \in [a,b]} x^2, \max_{x \in [a,b]} x^2] = [0, \max(a^2, b^2)] = [0, \norm{A}].
    \end{equation}

    \item \textbf{Proof of $\sigma(A) = [0, \norm{A}]$:}
    The operator $A - \lambda I$ is invertible if and only if the function $g_\lambda(x) = \frac{1}{x^2 - \lambda}$ is in $\Linfdomain$.

    \begin{itemize}
        \item[$\circ$] \textbf{Case 1: $\lambda \in [0, \norm{A}]$ (The range $R$)}
        Since $\lambda \in [0, \max(a^2, b^2)]$, there exists $x_0 \in [a,b]$ such that $x_0^2 = \lambda$. Thus, $x^2 - \lambda$ vanishes at $x_0$. This means that $g_\lambda(x)$ is unbounded (not in $\Linfdomain$) in any neighborhood of $x_0$. Therefore, $A - \lambda I$ is not invertible, so $\lambda \in \sigma(A)$. This shows $R \subseteq \sigma(A)$.

        \item[$\circ$] \textbf{Case 2: $\lambda \notin [0, \norm{A}]$ (The resolvent set)}
        The value $\lambda$ is separated from the compact set $[0, \norm{A}]$ by a positive distance $\delta$. Let $\delta = \dist(\lambda, [0, \norm{A}]) > 0$.
        The inverse operator $R_\lambda = (A - \lambda I)^{-1}$ is multiplication by $g_\lambda(x)$. Since
        $$\abs{g_\lambda(x)} = \frac{1}{\abs{x^2 - \lambda}} \le \frac{1}{\delta}$$
        for all $x \in [a,b]$, the function $g_\lambda$ is bounded and thus $g_\lambda \in \Linfdomain$. This means the inverse operator $R_\lambda$ is bounded, so $A - \lambda I$ is invertible. Thus, $\lambda \notin \sigma(A)$.
    \end{itemize}
\end{enumerate}
Therefore, the spectrum of the multiplication operator $A$ is:
\begin{equation}
\sigma(A) = [0, \norm{A}] = [0, \max(a^2, b^2)].
\end{equation}

\end{document}